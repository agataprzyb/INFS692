% Options for packages loaded elsewhere
\PassOptionsToPackage{unicode}{hyperref}
\PassOptionsToPackage{hyphens}{url}
%
\documentclass[
]{article}
\usepackage{amsmath,amssymb}
\usepackage{lmodern}
\usepackage{iftex}
\ifPDFTeX
  \usepackage[T1]{fontenc}
  \usepackage[utf8]{inputenc}
  \usepackage{textcomp} % provide euro and other symbols
\else % if luatex or xetex
  \usepackage{unicode-math}
  \defaultfontfeatures{Scale=MatchLowercase}
  \defaultfontfeatures[\rmfamily]{Ligatures=TeX,Scale=1}
\fi
% Use upquote if available, for straight quotes in verbatim environments
\IfFileExists{upquote.sty}{\usepackage{upquote}}{}
\IfFileExists{microtype.sty}{% use microtype if available
  \usepackage[]{microtype}
  \UseMicrotypeSet[protrusion]{basicmath} % disable protrusion for tt fonts
}{}
\makeatletter
\@ifundefined{KOMAClassName}{% if non-KOMA class
  \IfFileExists{parskip.sty}{%
    \usepackage{parskip}
  }{% else
    \setlength{\parindent}{0pt}
    \setlength{\parskip}{6pt plus 2pt minus 1pt}}
}{% if KOMA class
  \KOMAoptions{parskip=half}}
\makeatother
\usepackage{xcolor}
\usepackage[margin=1in]{geometry}
\usepackage{color}
\usepackage{fancyvrb}
\newcommand{\VerbBar}{|}
\newcommand{\VERB}{\Verb[commandchars=\\\{\}]}
\DefineVerbatimEnvironment{Highlighting}{Verbatim}{commandchars=\\\{\}}
% Add ',fontsize=\small' for more characters per line
\usepackage{framed}
\definecolor{shadecolor}{RGB}{248,248,248}
\newenvironment{Shaded}{\begin{snugshade}}{\end{snugshade}}
\newcommand{\AlertTok}[1]{\textcolor[rgb]{0.94,0.16,0.16}{#1}}
\newcommand{\AnnotationTok}[1]{\textcolor[rgb]{0.56,0.35,0.01}{\textbf{\textit{#1}}}}
\newcommand{\AttributeTok}[1]{\textcolor[rgb]{0.77,0.63,0.00}{#1}}
\newcommand{\BaseNTok}[1]{\textcolor[rgb]{0.00,0.00,0.81}{#1}}
\newcommand{\BuiltInTok}[1]{#1}
\newcommand{\CharTok}[1]{\textcolor[rgb]{0.31,0.60,0.02}{#1}}
\newcommand{\CommentTok}[1]{\textcolor[rgb]{0.56,0.35,0.01}{\textit{#1}}}
\newcommand{\CommentVarTok}[1]{\textcolor[rgb]{0.56,0.35,0.01}{\textbf{\textit{#1}}}}
\newcommand{\ConstantTok}[1]{\textcolor[rgb]{0.00,0.00,0.00}{#1}}
\newcommand{\ControlFlowTok}[1]{\textcolor[rgb]{0.13,0.29,0.53}{\textbf{#1}}}
\newcommand{\DataTypeTok}[1]{\textcolor[rgb]{0.13,0.29,0.53}{#1}}
\newcommand{\DecValTok}[1]{\textcolor[rgb]{0.00,0.00,0.81}{#1}}
\newcommand{\DocumentationTok}[1]{\textcolor[rgb]{0.56,0.35,0.01}{\textbf{\textit{#1}}}}
\newcommand{\ErrorTok}[1]{\textcolor[rgb]{0.64,0.00,0.00}{\textbf{#1}}}
\newcommand{\ExtensionTok}[1]{#1}
\newcommand{\FloatTok}[1]{\textcolor[rgb]{0.00,0.00,0.81}{#1}}
\newcommand{\FunctionTok}[1]{\textcolor[rgb]{0.00,0.00,0.00}{#1}}
\newcommand{\ImportTok}[1]{#1}
\newcommand{\InformationTok}[1]{\textcolor[rgb]{0.56,0.35,0.01}{\textbf{\textit{#1}}}}
\newcommand{\KeywordTok}[1]{\textcolor[rgb]{0.13,0.29,0.53}{\textbf{#1}}}
\newcommand{\NormalTok}[1]{#1}
\newcommand{\OperatorTok}[1]{\textcolor[rgb]{0.81,0.36,0.00}{\textbf{#1}}}
\newcommand{\OtherTok}[1]{\textcolor[rgb]{0.56,0.35,0.01}{#1}}
\newcommand{\PreprocessorTok}[1]{\textcolor[rgb]{0.56,0.35,0.01}{\textit{#1}}}
\newcommand{\RegionMarkerTok}[1]{#1}
\newcommand{\SpecialCharTok}[1]{\textcolor[rgb]{0.00,0.00,0.00}{#1}}
\newcommand{\SpecialStringTok}[1]{\textcolor[rgb]{0.31,0.60,0.02}{#1}}
\newcommand{\StringTok}[1]{\textcolor[rgb]{0.31,0.60,0.02}{#1}}
\newcommand{\VariableTok}[1]{\textcolor[rgb]{0.00,0.00,0.00}{#1}}
\newcommand{\VerbatimStringTok}[1]{\textcolor[rgb]{0.31,0.60,0.02}{#1}}
\newcommand{\WarningTok}[1]{\textcolor[rgb]{0.56,0.35,0.01}{\textbf{\textit{#1}}}}
\usepackage{graphicx}
\makeatletter
\def\maxwidth{\ifdim\Gin@nat@width>\linewidth\linewidth\else\Gin@nat@width\fi}
\def\maxheight{\ifdim\Gin@nat@height>\textheight\textheight\else\Gin@nat@height\fi}
\makeatother
% Scale images if necessary, so that they will not overflow the page
% margins by default, and it is still possible to overwrite the defaults
% using explicit options in \includegraphics[width, height, ...]{}
\setkeys{Gin}{width=\maxwidth,height=\maxheight,keepaspectratio}
% Set default figure placement to htbp
\makeatletter
\def\fps@figure{htbp}
\makeatother
\setlength{\emergencystretch}{3em} % prevent overfull lines
\providecommand{\tightlist}{%
  \setlength{\itemsep}{0pt}\setlength{\parskip}{0pt}}
\setcounter{secnumdepth}{-\maxdimen} % remove section numbering
\ifLuaTeX
  \usepackage{selnolig}  % disable illegal ligatures
\fi
\IfFileExists{bookmark.sty}{\usepackage{bookmark}}{\usepackage{hyperref}}
\IfFileExists{xurl.sty}{\usepackage{xurl}}{} % add URL line breaks if available
\urlstyle{same} % disable monospaced font for URLs
\hypersetup{
  pdftitle={INFS 692 - Final Project Model 3},
  pdfauthor={Agata},
  hidelinks,
  pdfcreator={LaTeX via pandoc}}

\title{INFS 692 - Final Project Model 3}
\author{Agata}
\date{2022-12-14}

\begin{document}
\maketitle

\hypertarget{libraries}{%
\subsubsection{Libraries}\label{libraries}}

\begin{Shaded}
\begin{Highlighting}[]
\FunctionTok{library}\NormalTok{(latexpdf)}
\FunctionTok{library}\NormalTok{(rmarkdown)}
\FunctionTok{library}\NormalTok{(readr)}
\FunctionTok{library}\NormalTok{(plyr)}
\FunctionTok{library}\NormalTok{(dplyr)}
\end{Highlighting}
\end{Shaded}

\begin{verbatim}
## 
## Attaching package: 'dplyr'
\end{verbatim}

\begin{verbatim}
## The following objects are masked from 'package:plyr':
## 
##     arrange, count, desc, failwith, id, mutate, rename, summarise,
##     summarize
\end{verbatim}

\begin{verbatim}
## The following objects are masked from 'package:stats':
## 
##     filter, lag
\end{verbatim}

\begin{verbatim}
## The following objects are masked from 'package:base':
## 
##     intersect, setdiff, setequal, union
\end{verbatim}

\begin{Shaded}
\begin{Highlighting}[]
\FunctionTok{library}\NormalTok{(ggplot2)}
\FunctionTok{library}\NormalTok{(ggpubr)}
\end{Highlighting}
\end{Shaded}

\begin{verbatim}
## 
## Attaching package: 'ggpubr'
\end{verbatim}

\begin{verbatim}
## The following object is masked from 'package:plyr':
## 
##     mutate
\end{verbatim}

\begin{Shaded}
\begin{Highlighting}[]
\FunctionTok{library}\NormalTok{(gridExtra)}
\end{Highlighting}
\end{Shaded}

\begin{verbatim}
## 
## Attaching package: 'gridExtra'
\end{verbatim}

\begin{verbatim}
## The following object is masked from 'package:dplyr':
## 
##     combine
\end{verbatim}

\begin{Shaded}
\begin{Highlighting}[]
\FunctionTok{library}\NormalTok{(COUNT)}
\end{Highlighting}
\end{Shaded}

\begin{verbatim}
## Loading required package: msme
\end{verbatim}

\begin{verbatim}
## Loading required package: MASS
\end{verbatim}

\begin{verbatim}
## 
## Attaching package: 'MASS'
\end{verbatim}

\begin{verbatim}
## The following object is masked from 'package:dplyr':
## 
##     select
\end{verbatim}

\begin{verbatim}
## Loading required package: lattice
\end{verbatim}

\begin{verbatim}
## Loading required package: sandwich
\end{verbatim}

\begin{Shaded}
\begin{Highlighting}[]
\FunctionTok{library}\NormalTok{(caret)}
\FunctionTok{library}\NormalTok{(rstatix)}
\end{Highlighting}
\end{Shaded}

\begin{verbatim}
## 
## Attaching package: 'rstatix'
\end{verbatim}

\begin{verbatim}
## The following object is masked from 'package:MASS':
## 
##     select
\end{verbatim}

\begin{verbatim}
## The following objects are masked from 'package:plyr':
## 
##     desc, mutate
\end{verbatim}

\begin{verbatim}
## The following object is masked from 'package:stats':
## 
##     filter
\end{verbatim}

\begin{Shaded}
\begin{Highlighting}[]
\FunctionTok{library}\NormalTok{(modeldata)}
\FunctionTok{library}\NormalTok{(rsample)    }\CommentTok{\# for creating validation splits}
\FunctionTok{library}\NormalTok{(recipes)    }\CommentTok{\# for feature engineering}
\end{Highlighting}
\end{Shaded}

\begin{verbatim}
## 
## Attaching package: 'recipes'
\end{verbatim}

\begin{verbatim}
## The following object is masked from 'package:stats':
## 
##     step
\end{verbatim}

\begin{Shaded}
\begin{Highlighting}[]
\FunctionTok{library}\NormalTok{(purrr)      }\CommentTok{\#for mapping}
\end{Highlighting}
\end{Shaded}

\begin{verbatim}
## 
## Attaching package: 'purrr'
\end{verbatim}

\begin{verbatim}
## The following object is masked from 'package:caret':
## 
##     lift
\end{verbatim}

\begin{verbatim}
## The following object is masked from 'package:plyr':
## 
##     compact
\end{verbatim}

\begin{Shaded}
\begin{Highlighting}[]
\FunctionTok{library}\NormalTok{(tidyverse)  }\CommentTok{\# for filtering }
\end{Highlighting}
\end{Shaded}

\begin{verbatim}
## -- Attaching packages --------------------------------------- tidyverse 1.3.2 --
\end{verbatim}

\begin{verbatim}
## v tibble  3.1.8     v stringr 1.5.0
## v tidyr   1.2.1     v forcats 0.5.2
## -- Conflicts ------------------------------------------ tidyverse_conflicts() --
## x dplyr::arrange()     masks plyr::arrange()
## x gridExtra::combine() masks dplyr::combine()
## x purrr::compact()     masks plyr::compact()
## x dplyr::count()       masks plyr::count()
## x dplyr::failwith()    masks plyr::failwith()
## x rstatix::filter()    masks dplyr::filter(), stats::filter()
## x stringr::fixed()     masks recipes::fixed()
## x dplyr::id()          masks plyr::id()
## x dplyr::lag()         masks stats::lag()
## x purrr::lift()        masks caret::lift()
## x rstatix::mutate()    masks ggpubr::mutate(), dplyr::mutate(), plyr::mutate()
## x dplyr::rename()      masks plyr::rename()
## x rstatix::select()    masks MASS::select(), dplyr::select()
## x dplyr::summarise()   masks plyr::summarise()
## x dplyr::summarize()   masks plyr::summarize()
\end{verbatim}

\begin{Shaded}
\begin{Highlighting}[]
\FunctionTok{library}\NormalTok{(ROCR)      }\CommentTok{\# ROC Curves}
\FunctionTok{library}\NormalTok{(pROC)      }\CommentTok{\# ROC Curves}
\end{Highlighting}
\end{Shaded}

\begin{verbatim}
## Type 'citation("pROC")' for a citation.
## 
## Attaching package: 'pROC'
## 
## The following objects are masked from 'package:stats':
## 
##     cov, smooth, var
\end{verbatim}

\begin{Shaded}
\begin{Highlighting}[]
\FunctionTok{library}\NormalTok{(rpart)      }\CommentTok{\# decision tree application}
\FunctionTok{library}\NormalTok{(rpart.plot)  }\CommentTok{\# plotting decision trees}
\FunctionTok{library}\NormalTok{(vip)         }\CommentTok{\# for feature importance}
\end{Highlighting}
\end{Shaded}

\begin{verbatim}
## 
## Attaching package: 'vip'
## 
## The following object is masked from 'package:utils':
## 
##     vi
\end{verbatim}

\begin{Shaded}
\begin{Highlighting}[]
\FunctionTok{library}\NormalTok{(pdp) }
\end{Highlighting}
\end{Shaded}

\begin{verbatim}
## 
## Attaching package: 'pdp'
## 
## The following object is masked from 'package:purrr':
## 
##     partial
\end{verbatim}

\begin{Shaded}
\begin{Highlighting}[]
\FunctionTok{library}\NormalTok{(stringr)     }\CommentTok{\# for string functionality}

\CommentTok{\# Modeling packages}
\FunctionTok{library}\NormalTok{(tidyverse)  }\CommentTok{\# data manipulation}
\FunctionTok{library}\NormalTok{(cluster)     }\CommentTok{\# for general clustering algorithms}
\FunctionTok{library}\NormalTok{(factoextra)  }\CommentTok{\# for visualizing cluster results}
\end{Highlighting}
\end{Shaded}

\begin{verbatim}
## Welcome! Want to learn more? See two factoextra-related books at https://goo.gl/ve3WBa
\end{verbatim}

\begin{Shaded}
\begin{Highlighting}[]
\CommentTok{\# Modeling packages}
\FunctionTok{library}\NormalTok{(mclust)   }\CommentTok{\# for fitting clustering algorithms}
\end{Highlighting}
\end{Shaded}

\begin{verbatim}
## Package 'mclust' version 6.0.0
## Type 'citation("mclust")' for citing this R package in publications.
## 
## Attaching package: 'mclust'
## 
## The following object is masked from 'package:purrr':
## 
##     map
\end{verbatim}

\hypertarget{importing-data}{%
\subsubsection{Importing Data}\label{importing-data}}

\begin{Shaded}
\begin{Highlighting}[]
\NormalTok{data1 }\OtherTok{\textless{}{-}} \FunctionTok{read.csv}\NormalTok{(}\StringTok{"radiomics\_completedata.csv"}\NormalTok{, }\AttributeTok{sep =} \StringTok{","}\NormalTok{)}
\end{Highlighting}
\end{Shaded}

\hypertarget{preprocessing-data}{%
\subsubsection{Preprocessing data}\label{preprocessing-data}}

Since we're not taking into consideration categorical variables or
Failure.binary, I decided to split the data so that it only takes into
consideration variables with Entropy. I thought this split had the most
interesting results.

\begin{Shaded}
\begin{Highlighting}[]
\CommentTok{\#Check for null and missing values}

\FunctionTok{which}\NormalTok{(}\FunctionTok{is.null}\NormalTok{(data1))}
\end{Highlighting}
\end{Shaded}

\begin{verbatim}
## integer(0)
\end{verbatim}

\begin{Shaded}
\begin{Highlighting}[]
\FunctionTok{which}\NormalTok{(}\FunctionTok{is.na}\NormalTok{(data1))}
\end{Highlighting}
\end{Shaded}

\begin{verbatim}
## integer(0)
\end{verbatim}

\begin{Shaded}
\begin{Highlighting}[]
\CommentTok{\#Data split}


\NormalTok{sub2 }\OtherTok{\textless{}{-}} \FunctionTok{subset}\NormalTok{(data1, }\AttributeTok{select=} \SpecialCharTok{{-}}\FunctionTok{c}\NormalTok{(Institution, Failure.binary, Failure))}


\NormalTok{sub2 }\OtherTok{\textless{}{-}}\NormalTok{ dplyr}\SpecialCharTok{::}\FunctionTok{select}\NormalTok{(sub2, }\FunctionTok{contains}\NormalTok{(}\StringTok{"Entropy"}\NormalTok{))}




\CommentTok{\#Check for normality}

\FunctionTok{hist}\NormalTok{(sub2}\SpecialCharTok{$}\NormalTok{Entropy\_cooc.W.ADC)}
\end{Highlighting}
\end{Shaded}

\includegraphics{INFS-692---Final-Project-Model-3_files/figure-latex/unnamed-chunk-3-1.pdf}

\begin{Shaded}
\begin{Highlighting}[]
\CommentTok{\# or}


\NormalTok{sub1shapiro }\OtherTok{\textless{}{-}} \FunctionTok{shapiro.test}\NormalTok{(sub2}\SpecialCharTok{$}\NormalTok{Entropy\_cooc.W.ADC)}
\NormalTok{sub1shapiro}
\end{Highlighting}
\end{Shaded}

\begin{verbatim}
## 
##  Shapiro-Wilk normality test
## 
## data:  sub2$Entropy_cooc.W.ADC
## W = 0.98903, p-value = 0.135
\end{verbatim}

\begin{Shaded}
\begin{Highlighting}[]
\CommentTok{\#based on the histogram, the data is not normalized.This is also enhanced by the shapiro test}
\CommentTok{\#where the p{-}value is \textless{} 0.05, which means that the data is not normalized.}


\CommentTok{\#Normalize Data}

\NormalTok{scale\_data }\OtherTok{\textless{}{-}}  \FunctionTok{as.data.frame}\NormalTok{(}\FunctionTok{scale}\NormalTok{(sub2, }\AttributeTok{center =} \ConstantTok{TRUE}\NormalTok{, }\AttributeTok{scale =} \ConstantTok{TRUE}\NormalTok{))}

\FunctionTok{summary}\NormalTok{(scale\_data}\SpecialCharTok{$}\NormalTok{Entropy\_cooc.W.ADC)}
\end{Highlighting}
\end{Shaded}

\begin{verbatim}
##       Min.    1st Qu.     Median       Mean    3rd Qu.       Max. 
## -2.6407173 -0.6921994  0.0001827  0.0000000  0.6719746  2.1464095
\end{verbatim}

\begin{Shaded}
\begin{Highlighting}[]
\FunctionTok{sd}\NormalTok{(scale\_data}\SpecialCharTok{$}\NormalTok{Entropy\_cooc.W.ADC)}
\end{Highlighting}
\end{Shaded}

\begin{verbatim}
## [1] 1
\end{verbatim}

\begin{Shaded}
\begin{Highlighting}[]
\CommentTok{\#Now the data has a mean of 0 and a standard deviation of 1, meaning the data is normalized. }


\CommentTok{\#check correlation for full data set except categorical variables}

\NormalTok{cor1 }\OtherTok{\textless{}{-}} \FunctionTok{cor}\NormalTok{(scale\_data)}
\CommentTok{\#cor1 commented out for pdf page saving purposes}
\end{Highlighting}
\end{Shaded}

\hypertarget{k-means}{%
\subsubsection{K-Means}\label{k-means}}

Based on the optimal number of clusters graph below, the optimal number
is k=2, which is why the k-means clustering is set to 2.

\begin{Shaded}
\begin{Highlighting}[]
\NormalTok{df }\OtherTok{\textless{}{-}}\NormalTok{ scale\_data}


\CommentTok{\#Determining Optimal Number of Clusters}
\FunctionTok{set.seed}\NormalTok{(}\DecValTok{123}\NormalTok{)}

\CommentTok{\#function to compute total within{-}cluster sum of square }
\NormalTok{wss }\OtherTok{\textless{}{-}} \ControlFlowTok{function}\NormalTok{(k) \{}
  \FunctionTok{kmeans}\NormalTok{(df, k, }\AttributeTok{nstart =} \DecValTok{10}\NormalTok{ )}\SpecialCharTok{$}\NormalTok{tot.withinss}
\NormalTok{\}}

\CommentTok{\# Compute and plot wss for k = 1 to k = 15}
\NormalTok{k.values }\OtherTok{\textless{}{-}} \DecValTok{1}\SpecialCharTok{:}\DecValTok{15}

\CommentTok{\# extract wss for 2{-}15 clusters}
\NormalTok{wss\_values }\OtherTok{\textless{}{-}} \FunctionTok{map\_dbl}\NormalTok{(k.values, wss)}

\CommentTok{\#or use this}
\FunctionTok{fviz\_nbclust}\NormalTok{(df, kmeans, }\AttributeTok{method =} \StringTok{"silhouette"}\NormalTok{)}
\end{Highlighting}
\end{Shaded}

\includegraphics{INFS-692---Final-Project-Model-3_files/figure-latex/unnamed-chunk-4-1.pdf}

\begin{Shaded}
\begin{Highlighting}[]
\CommentTok{\# Compute k{-}means clustering with k = 2}
\FunctionTok{set.seed}\NormalTok{(}\DecValTok{123}\NormalTok{)}
\NormalTok{final }\OtherTok{\textless{}{-}} \FunctionTok{kmeans}\NormalTok{(df, }\DecValTok{2}\NormalTok{, }\AttributeTok{nstart =} \DecValTok{25}\NormalTok{)}
\FunctionTok{print}\NormalTok{(final)}
\end{Highlighting}
\end{Shaded}

\begin{verbatim}
## K-means clustering with 2 clusters of sizes 50, 147
## 
## Cluster means:
##   Entropy_cooc.W.ADC Entropy_hist.PET Entropy_cooc.L.PET Entropy_align.L.PET
## 1         0.04845450        1.5003007          1.6813985           1.6880661
## 2        -0.01648112       -0.5103064         -0.5719043          -0.5741722
##   Entropy_area.L.PET Entropy_cooc.H.PET Entropy_align.H.PET Entropy_area.H.PET
## 1          1.6893793          1.4404122            1.550297          1.6279234
## 2         -0.5746188         -0.4899361           -0.527312         -0.5537154
##   Entropy_cooc.W.PET Entropy_align.W.PET Entropy_area.W.PET Entropy_hist.ADC
## 1          1.4784780           1.5543465          1.6167770        1.6284344
## 2         -0.5028837          -0.5286893         -0.5499242       -0.5538893
##   Entropy_cooc.L.ADC Entropy_align.L.ADC Entropy_area.L.ADC Entropy_cooc.H.ADC
## 1          1.6827114           1.6982212          1.7010816          1.7011475
## 2         -0.5723508          -0.5776262         -0.5785992         -0.5786216
##   Entropy_align.H.ADC Entropy_area.H.ADC Entropy_align.W.ADC Entropy_area.W.ADC
## 1           1.7093530          1.7066118            1.661714           1.672228
## 2          -0.5814126         -0.5804802           -0.565209          -0.568785
## 
## Clustering vector:
##   1   2   3   4   5   6   7   8   9  10  11  12  13  14  15  16  17  18  19  20 
##   2   2   2   2   2   2   2   2   2   2   2   2   2   2   2   2   2   2   2   2 
##  21  22  23  24  25  26  27  28  29  30  31  32  33  34  35  36  37  38  39  40 
##   2   2   2   2   2   2   2   2   2   2   2   2   2   2   2   2   2   2   2   2 
##  41  42  43  44  45  46  47  48  49  50  51  52  53  54  55  56  57  58  59  60 
##   2   2   2   2   2   2   2   2   2   2   2   2   2   2   2   2   2   2   2   2 
##  61  62  63  64  65  66  67  68  69  70  71  72  73  74  75  76  77  78  79  80 
##   2   2   2   2   2   2   2   2   2   2   2   2   2   2   2   2   2   2   2   2 
##  81  82  83  84  85  86  87  88  89  90  91  92  93  94  95  96  97  98  99 100 
##   2   2   2   2   2   2   2   2   2   2   2   2   2   2   2   2   2   2   2   2 
## 101 102 103 104 105 106 107 108 109 110 111 112 113 114 115 116 117 118 119 120 
##   2   2   2   2   2   2   2   2   2   2   2   2   2   2   2   2   2   2   2   2 
## 121 122 123 124 125 126 127 128 129 130 131 132 133 134 135 136 137 138 139 140 
##   2   2   2   2   2   2   2   2   2   2   2   2   2   2   2   2   2   2   2   2 
## 141 142 143 144 145 146 147 148 149 150 151 152 153 154 155 156 157 158 159 160 
##   2   2   2   2   2   2   2   1   1   1   1   1   1   1   1   1   1   1   1   1 
## 161 162 163 164 165 166 167 168 169 170 171 172 173 174 175 176 177 178 179 180 
##   1   1   1   1   1   1   1   1   1   1   1   1   1   1   1   1   1   1   1   1 
## 181 182 183 184 185 186 187 188 189 190 191 192 193 194 195 196 197 
##   1   1   1   1   1   1   1   1   1   1   1   1   1   1   1   1   1 
## 
## Within cluster sum of squares by cluster:
## [1] 224.1776 300.2509
##  (between_SS / total_SS =  86.6 %)
## 
## Available components:
## 
## [1] "cluster"      "centers"      "totss"        "withinss"     "tot.withinss"
## [6] "betweenss"    "size"         "iter"         "ifault"
\end{verbatim}

\begin{Shaded}
\begin{Highlighting}[]
\CommentTok{\#final data}
\FunctionTok{fviz\_cluster}\NormalTok{(final, }\AttributeTok{data =}\NormalTok{ df)}
\end{Highlighting}
\end{Shaded}

\includegraphics{INFS-692---Final-Project-Model-3_files/figure-latex/unnamed-chunk-4-2.pdf}
\#\#\# Hierarchical

\begin{Shaded}
\begin{Highlighting}[]
\CommentTok{\# For reproducibility}
\FunctionTok{set.seed}\NormalTok{(}\DecValTok{123}\NormalTok{)}

\CommentTok{\# Dissimilarity matrix}
\NormalTok{d }\OtherTok{\textless{}{-}} \FunctionTok{dist}\NormalTok{(df, }\AttributeTok{method =} \StringTok{"euclidean"}\NormalTok{)}

\CommentTok{\# Hierarchical clustering using Complete Linkage}
\NormalTok{hc1 }\OtherTok{\textless{}{-}} \FunctionTok{hclust}\NormalTok{(d, }\AttributeTok{method =} \StringTok{"complete"}\NormalTok{ )}

\CommentTok{\# For reproducibility}
\FunctionTok{set.seed}\NormalTok{(}\DecValTok{123}\NormalTok{)}

\CommentTok{\# Compute maximum or complete linkage clustering with agnes}
\NormalTok{hc2 }\OtherTok{\textless{}{-}} \FunctionTok{agnes}\NormalTok{(df, }\AttributeTok{method =} \StringTok{"complete"}\NormalTok{)}

\CommentTok{\# Agglomerative coefficient}
\NormalTok{hc2}\SpecialCharTok{$}\NormalTok{ac}
\end{Highlighting}
\end{Shaded}

\begin{verbatim}
## [1] 0.9471993
\end{verbatim}

\begin{Shaded}
\begin{Highlighting}[]
\CommentTok{\# methods to assess}
\NormalTok{m }\OtherTok{\textless{}{-}} \FunctionTok{c}\NormalTok{( }\StringTok{"average"}\NormalTok{, }\StringTok{"single"}\NormalTok{, }\StringTok{"complete"}\NormalTok{, }\StringTok{"ward"}\NormalTok{)}
\FunctionTok{names}\NormalTok{(m) }\OtherTok{\textless{}{-}} \FunctionTok{c}\NormalTok{( }\StringTok{"average"}\NormalTok{, }\StringTok{"single"}\NormalTok{, }\StringTok{"complete"}\NormalTok{, }\StringTok{"ward"}\NormalTok{)}

\CommentTok{\# function to compute coefficient}
\NormalTok{ac }\OtherTok{\textless{}{-}} \ControlFlowTok{function}\NormalTok{(x) \{}
  \FunctionTok{agnes}\NormalTok{(df, }\AttributeTok{method =}\NormalTok{ x)}\SpecialCharTok{$}\NormalTok{ac}
\NormalTok{\}}

\CommentTok{\# get agglomerative coefficient for each linkage method}
\NormalTok{purrr}\SpecialCharTok{::}\FunctionTok{map\_dbl}\NormalTok{(m, ac)}
\end{Highlighting}
\end{Shaded}

\begin{verbatim}
##   average    single  complete      ward 
## 0.9251753 0.9063889 0.9471993 0.9908626
\end{verbatim}

\begin{Shaded}
\begin{Highlighting}[]
\CommentTok{\# compute divisive hierarchical clustering}
\NormalTok{hc4 }\OtherTok{\textless{}{-}} \FunctionTok{diana}\NormalTok{(df)}

\CommentTok{\# Divise coefficient; amount of clustering structure found}
\NormalTok{hc4}\SpecialCharTok{$}\NormalTok{dc}
\end{Highlighting}
\end{Shaded}

\begin{verbatim}
## [1] 0.9420499
\end{verbatim}

\begin{Shaded}
\begin{Highlighting}[]
\CommentTok{\# plots to compare}
\NormalTok{p1 }\OtherTok{\textless{}{-}} \FunctionTok{fviz\_nbclust}\NormalTok{(df, }\AttributeTok{FUN =}\NormalTok{ hcut, }\AttributeTok{method =} \StringTok{"wss"}\NormalTok{,}
                   \AttributeTok{k.max =} \DecValTok{10}\NormalTok{) }\SpecialCharTok{+}
  \FunctionTok{ggtitle}\NormalTok{(}\StringTok{"(A) Elbow method"}\NormalTok{)}
\NormalTok{p2 }\OtherTok{\textless{}{-}} \FunctionTok{fviz\_nbclust}\NormalTok{(df, }\AttributeTok{FUN =}\NormalTok{ hcut, }\AttributeTok{method =} \StringTok{"silhouette"}\NormalTok{,}
                   \AttributeTok{k.max =} \DecValTok{10}\NormalTok{) }\SpecialCharTok{+}
  \FunctionTok{ggtitle}\NormalTok{(}\StringTok{"(B) Silhouette method"}\NormalTok{)}
\NormalTok{p3 }\OtherTok{\textless{}{-}} \FunctionTok{fviz\_nbclust}\NormalTok{(df, }\AttributeTok{FUN =}\NormalTok{ hcut, }\AttributeTok{method =} \StringTok{"gap\_stat"}\NormalTok{,}
                   \AttributeTok{k.max =} \DecValTok{10}\NormalTok{) }\SpecialCharTok{+}
  \FunctionTok{ggtitle}\NormalTok{(}\StringTok{"(A) Gap Statistics method"}\NormalTok{)}

\NormalTok{gridExtra}\SpecialCharTok{::}\FunctionTok{grid.arrange}\NormalTok{(p1, p2, p3, }\AttributeTok{nrow =} \DecValTok{1}\NormalTok{)}
\end{Highlighting}
\end{Shaded}

\includegraphics{INFS-692---Final-Project-Model-3_files/figure-latex/unnamed-chunk-5-1.pdf}

\begin{Shaded}
\begin{Highlighting}[]
\CommentTok{\# Ward\textquotesingle{}s method}

\NormalTok{hc5 }\OtherTok{\textless{}{-}} \FunctionTok{hclust}\NormalTok{(d, }\AttributeTok{method =} \StringTok{"ward.D2"}\NormalTok{ )}

\NormalTok{dend\_plot }\OtherTok{\textless{}{-}} \FunctionTok{fviz\_dend}\NormalTok{(hc5)}
\end{Highlighting}
\end{Shaded}

\begin{verbatim}
## Warning: The `<scale>` argument of `guides()` cannot be `FALSE`. Use "none" instead as
## of ggplot2 3.3.4.
## i The deprecated feature was likely used in the factoextra package.
##   Please report the issue at <]8;;https://github.com/kassambara/factoextra/issueshttps://github.com/kassambara/factoextra/issues]8;;>.
\end{verbatim}

\begin{Shaded}
\begin{Highlighting}[]
\NormalTok{dend\_data }\OtherTok{\textless{}{-}} \FunctionTok{attr}\NormalTok{(dend\_plot, }\StringTok{"dendrogram"}\NormalTok{)}
\NormalTok{dend\_cuts }\OtherTok{\textless{}{-}} \FunctionTok{cut}\NormalTok{(dend\_data, }\AttributeTok{h =} \DecValTok{8}\NormalTok{)}
\FunctionTok{fviz\_dend}\NormalTok{(dend\_cuts}\SpecialCharTok{$}\NormalTok{lower[[}\DecValTok{2}\NormalTok{]])}
\end{Highlighting}
\end{Shaded}

\includegraphics{INFS-692---Final-Project-Model-3_files/figure-latex/unnamed-chunk-5-2.pdf}
As shown above, both silhouette and gap-statistics show the same optimal
k number of clusters. Which is why we remain at k = 2.

\begin{Shaded}
\begin{Highlighting}[]
\CommentTok{\# Cut tree into 3 groups}
\NormalTok{sub\_grp }\OtherTok{\textless{}{-}} \FunctionTok{cutree}\NormalTok{(hc5, }\AttributeTok{k =} \DecValTok{2}\NormalTok{)}

\CommentTok{\# Number of members in each cluster}
\FunctionTok{table}\NormalTok{(sub\_grp)}
\end{Highlighting}
\end{Shaded}

\begin{verbatim}
## sub_grp
##   1   2 
## 147  50
\end{verbatim}

\begin{Shaded}
\begin{Highlighting}[]
\CommentTok{\# Plot full dendogram}
\FunctionTok{fviz\_dend}\NormalTok{(}
\NormalTok{  hc5,}
  \AttributeTok{k =} \DecValTok{2}\NormalTok{,}
  \AttributeTok{horiz =} \ConstantTok{TRUE}\NormalTok{,}
  \AttributeTok{rect =} \ConstantTok{TRUE}\NormalTok{,}
  \AttributeTok{rect\_fill =} \ConstantTok{TRUE}\NormalTok{,}
  \AttributeTok{rect\_border =} \StringTok{"jco"}\NormalTok{,}
  \AttributeTok{k\_colors =} \StringTok{"jco"}\NormalTok{,}
  \AttributeTok{cex =} \FloatTok{0.1}
\NormalTok{)}
\end{Highlighting}
\end{Shaded}

\includegraphics{INFS-692---Final-Project-Model-3_files/figure-latex/unnamed-chunk-6-1.pdf}

\begin{Shaded}
\begin{Highlighting}[]
\NormalTok{dend\_plot }\OtherTok{\textless{}{-}} \FunctionTok{fviz\_dend}\NormalTok{(hc5)                }\CommentTok{\# create full dendogram}
\NormalTok{dend\_data }\OtherTok{\textless{}{-}} \FunctionTok{attr}\NormalTok{(dend\_plot, }\StringTok{"dendrogram"}\NormalTok{) }\CommentTok{\# extract plot info}
\NormalTok{dend\_cuts }\OtherTok{\textless{}{-}} \FunctionTok{cut}\NormalTok{(dend\_data, }\AttributeTok{h =} \FloatTok{70.5}\NormalTok{)      }\CommentTok{\# cut the dendogram at }
\CommentTok{\# designated height}
\CommentTok{\# Create sub dendrogram plots}
\NormalTok{p1 }\OtherTok{\textless{}{-}} \FunctionTok{fviz\_dend}\NormalTok{(dend\_cuts}\SpecialCharTok{$}\NormalTok{lower[[}\DecValTok{1}\NormalTok{]])}
\NormalTok{p2 }\OtherTok{\textless{}{-}} \FunctionTok{fviz\_dend}\NormalTok{(dend\_cuts}\SpecialCharTok{$}\NormalTok{lower[[}\DecValTok{1}\NormalTok{]], }\AttributeTok{type =} \StringTok{\textquotesingle{}circular\textquotesingle{}}\NormalTok{)}

\CommentTok{\# Side by side plots}
\NormalTok{gridExtra}\SpecialCharTok{::}\FunctionTok{grid.arrange}\NormalTok{(p1, p2, }\AttributeTok{nrow =} \DecValTok{1}\NormalTok{)}
\end{Highlighting}
\end{Shaded}

\includegraphics{INFS-692---Final-Project-Model-3_files/figure-latex/unnamed-chunk-6-2.pdf}
Based on this clustering algorithm, the final output of the dendrogram
doesn't look as nice without zooming in. However, it does show more
details than K-Means.

\hypertarget{model-based}{%
\subsubsection{Model-Based}\label{model-based}}

\begin{Shaded}
\begin{Highlighting}[]
\NormalTok{my\_basket }\OtherTok{\textless{}{-}}\NormalTok{ scale\_data}

\CommentTok{\# Apply GMM model with 3 components}
\NormalTok{data1\_mc }\OtherTok{\textless{}{-}} \FunctionTok{Mclust}\NormalTok{(scale\_data, }\AttributeTok{G =} \DecValTok{3}\NormalTok{)}

\CommentTok{\# Plot results}

\FunctionTok{par}\NormalTok{(}\AttributeTok{mar=}\FunctionTok{c}\NormalTok{(}\DecValTok{1}\NormalTok{,}\DecValTok{1}\NormalTok{,}\DecValTok{1}\NormalTok{,}\DecValTok{1}\NormalTok{))}


\FunctionTok{plot}\NormalTok{(data1\_mc, }\AttributeTok{what =} \StringTok{"density"}\NormalTok{)}
\end{Highlighting}
\end{Shaded}

\includegraphics{INFS-692---Final-Project-Model-3_files/figure-latex/unnamed-chunk-7-1.pdf}

\begin{Shaded}
\begin{Highlighting}[]
\FunctionTok{plot}\NormalTok{(data1\_mc, }\AttributeTok{what =} \StringTok{"uncertainty"}\NormalTok{)}
\end{Highlighting}
\end{Shaded}

\includegraphics{INFS-692---Final-Project-Model-3_files/figure-latex/unnamed-chunk-7-2.pdf}

\begin{Shaded}
\begin{Highlighting}[]
\CommentTok{\# Observations with high uncertainty}
\FunctionTok{sort}\NormalTok{(data1\_mc}\SpecialCharTok{$}\NormalTok{uncertainty, }\AttributeTok{decreasing =} \ConstantTok{TRUE}\NormalTok{) }\SpecialCharTok{\%\textgreater{}\%} \FunctionTok{head}\NormalTok{()}
\end{Highlighting}
\end{Shaded}

\begin{verbatim}
##        33        38        26        70        72         3 
## 0.3812836 0.3201442 0.2158333 0.1526643 0.1145401 0.1145004
\end{verbatim}

\begin{Shaded}
\begin{Highlighting}[]
\NormalTok{data1\_optimal\_mc }\OtherTok{\textless{}{-}} \FunctionTok{Mclust}\NormalTok{(scale\_data)}


\NormalTok{legend\_args }\OtherTok{\textless{}{-}} \FunctionTok{list}\NormalTok{(}\AttributeTok{x =} \StringTok{"bottomright"}\NormalTok{, }\AttributeTok{ncol =} \DecValTok{5}\NormalTok{)}
\FunctionTok{plot}\NormalTok{(data1\_optimal\_mc, }\AttributeTok{what =} \StringTok{\textquotesingle{}BIC\textquotesingle{}}\NormalTok{, }\AttributeTok{legendArgs =}\NormalTok{ legend\_args)}
\end{Highlighting}
\end{Shaded}

\includegraphics{INFS-692---Final-Project-Model-3_files/figure-latex/unnamed-chunk-7-3.pdf}

\begin{Shaded}
\begin{Highlighting}[]
\FunctionTok{plot}\NormalTok{(data1\_optimal\_mc, }\AttributeTok{what =} \StringTok{\textquotesingle{}classification\textquotesingle{}}\NormalTok{)}
\end{Highlighting}
\end{Shaded}

\includegraphics{INFS-692---Final-Project-Model-3_files/figure-latex/unnamed-chunk-7-4.pdf}

\begin{Shaded}
\begin{Highlighting}[]
\FunctionTok{plot}\NormalTok{(data1\_optimal\_mc, }\AttributeTok{what =} \StringTok{\textquotesingle{}uncertainty\textquotesingle{}}\NormalTok{)}
\end{Highlighting}
\end{Shaded}

\includegraphics{INFS-692---Final-Project-Model-3_files/figure-latex/unnamed-chunk-7-5.pdf}

\begin{Shaded}
\begin{Highlighting}[]
\NormalTok{my\_basket\_mc }\OtherTok{\textless{}{-}} \FunctionTok{Mclust}\NormalTok{(my\_basket, }\DecValTok{1}\SpecialCharTok{:}\DecValTok{20}\NormalTok{)}


\FunctionTok{plot}\NormalTok{(my\_basket\_mc, }\AttributeTok{what =} \StringTok{\textquotesingle{}BIC\textquotesingle{}}\NormalTok{, }
     \AttributeTok{legendArgs =} \FunctionTok{list}\NormalTok{(}\AttributeTok{x =} \StringTok{"bottomright"}\NormalTok{, }\AttributeTok{ncol =} \DecValTok{5}\NormalTok{))}
\end{Highlighting}
\end{Shaded}

\includegraphics{INFS-692---Final-Project-Model-3_files/figure-latex/unnamed-chunk-7-6.pdf}

\begin{Shaded}
\begin{Highlighting}[]
\NormalTok{probabilities }\OtherTok{\textless{}{-}}\NormalTok{ my\_basket\_mc}\SpecialCharTok{$}\NormalTok{z }
\FunctionTok{colnames}\NormalTok{(probabilities) }\OtherTok{\textless{}{-}} \FunctionTok{paste0}\NormalTok{(}\StringTok{\textquotesingle{}C\textquotesingle{}}\NormalTok{, }\DecValTok{1}\SpecialCharTok{:}\DecValTok{3}\NormalTok{)}

\NormalTok{probabilities }\OtherTok{\textless{}{-}}\NormalTok{ probabilities }\SpecialCharTok{\%\textgreater{}\%}
  \FunctionTok{as.data.frame}\NormalTok{() }\SpecialCharTok{\%\textgreater{}\%}
  \FunctionTok{mutate}\NormalTok{(}\AttributeTok{id =} \FunctionTok{row\_number}\NormalTok{()) }\SpecialCharTok{\%\textgreater{}\%}
\NormalTok{  tidyr}\SpecialCharTok{::}\FunctionTok{gather}\NormalTok{(cluster, probability, }\SpecialCharTok{{-}}\NormalTok{id)}

\FunctionTok{ggplot}\NormalTok{(probabilities, }\FunctionTok{aes}\NormalTok{(probability)) }\SpecialCharTok{+}
  \FunctionTok{geom\_histogram}\NormalTok{() }\SpecialCharTok{+}
  \FunctionTok{facet\_wrap}\NormalTok{(}\SpecialCharTok{\textasciitilde{}}\NormalTok{ cluster, }\AttributeTok{nrow =} \DecValTok{2}\NormalTok{)}
\end{Highlighting}
\end{Shaded}

\begin{verbatim}
## `stat_bin()` using `bins = 30`. Pick better value with `binwidth`.
\end{verbatim}

\includegraphics{INFS-692---Final-Project-Model-3_files/figure-latex/unnamed-chunk-7-7.pdf}

\begin{Shaded}
\begin{Highlighting}[]
\NormalTok{uncertainty }\OtherTok{\textless{}{-}} \FunctionTok{data.frame}\NormalTok{(}
  \AttributeTok{id =} \DecValTok{1}\SpecialCharTok{:}\FunctionTok{nrow}\NormalTok{(my\_basket),}
  \AttributeTok{cluster =}\NormalTok{ my\_basket\_mc}\SpecialCharTok{$}\NormalTok{classification,}
  \AttributeTok{uncertainty =}\NormalTok{ my\_basket\_mc}\SpecialCharTok{$}\NormalTok{uncertainty}
\NormalTok{)}

\NormalTok{uncertainty }\SpecialCharTok{\%\textgreater{}\%}
  \FunctionTok{group\_by}\NormalTok{(cluster) }\SpecialCharTok{\%\textgreater{}\%}
  \FunctionTok{filter}\NormalTok{(uncertainty }\SpecialCharTok{\textgreater{}} \FloatTok{0.25}\NormalTok{) }\SpecialCharTok{\%\textgreater{}\%}
  \FunctionTok{ggplot}\NormalTok{(}\FunctionTok{aes}\NormalTok{(uncertainty, }\FunctionTok{reorder}\NormalTok{(id, uncertainty))) }\SpecialCharTok{+}
  \FunctionTok{geom\_point}\NormalTok{() }\SpecialCharTok{+}
  \FunctionTok{facet\_wrap}\NormalTok{(}\SpecialCharTok{\textasciitilde{}}\NormalTok{ cluster, }\AttributeTok{scales =} \StringTok{\textquotesingle{}free\_y\textquotesingle{}}\NormalTok{, }\AttributeTok{nrow =} \DecValTok{1}\NormalTok{)}
\end{Highlighting}
\end{Shaded}

\includegraphics{INFS-692---Final-Project-Model-3_files/figure-latex/unnamed-chunk-7-8.pdf}

\begin{Shaded}
\begin{Highlighting}[]
\NormalTok{cluster2 }\OtherTok{\textless{}{-}}\NormalTok{ my\_basket }\SpecialCharTok{\%\textgreater{}\%}
  \FunctionTok{scale}\NormalTok{() }\SpecialCharTok{\%\textgreater{}\%}
  \FunctionTok{as.data.frame}\NormalTok{() }\SpecialCharTok{\%\textgreater{}\%}
  \FunctionTok{mutate}\NormalTok{(}\AttributeTok{cluster =}\NormalTok{ my\_basket\_mc}\SpecialCharTok{$}\NormalTok{classification) }\SpecialCharTok{\%\textgreater{}\%}
  \FunctionTok{filter}\NormalTok{(cluster }\SpecialCharTok{==} \DecValTok{2}\NormalTok{) }\SpecialCharTok{\%\textgreater{}\%}
  \FunctionTok{select}\NormalTok{(}\SpecialCharTok{{-}}\NormalTok{cluster)}

\NormalTok{cluster2 }\SpecialCharTok{\%\textgreater{}\%}
\NormalTok{  tidyr}\SpecialCharTok{::}\FunctionTok{gather}\NormalTok{(product, std\_count) }\SpecialCharTok{\%\textgreater{}\%}
  \FunctionTok{group\_by}\NormalTok{(product) }\SpecialCharTok{\%\textgreater{}\%}
  \FunctionTok{summarize}\NormalTok{(}\AttributeTok{avg =} \FunctionTok{mean}\NormalTok{(std\_count)) }\SpecialCharTok{\%\textgreater{}\%}
  \FunctionTok{ggplot}\NormalTok{(}\FunctionTok{aes}\NormalTok{(avg, }\FunctionTok{reorder}\NormalTok{(product, avg))) }\SpecialCharTok{+}
  \FunctionTok{geom\_point}\NormalTok{() }\SpecialCharTok{+}
  \FunctionTok{labs}\NormalTok{(}\AttributeTok{x =} \StringTok{"Average Entropy"}\NormalTok{, }\AttributeTok{y =} \ConstantTok{NULL}\NormalTok{)}
\end{Highlighting}
\end{Shaded}

\includegraphics{INFS-692---Final-Project-Model-3_files/figure-latex/unnamed-chunk-7-9.pdf}
\#\#\# Conclusion

K-Means and Hierarchical models did not change their k cluster
numbering, but model-based did with k = 3. Each model has an interesting
way of displaying clusters, and the most interesting one of the three is
the Hierarchical one with the dendrogram.

\end{document}
