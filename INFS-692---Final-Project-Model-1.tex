% Options for packages loaded elsewhere
\PassOptionsToPackage{unicode}{hyperref}
\PassOptionsToPackage{hyphens}{url}
%
\documentclass[
]{article}
\usepackage{amsmath,amssymb}
\usepackage{lmodern}
\usepackage{iftex}
\ifPDFTeX
  \usepackage[T1]{fontenc}
  \usepackage[utf8]{inputenc}
  \usepackage{textcomp} % provide euro and other symbols
\else % if luatex or xetex
  \usepackage{unicode-math}
  \defaultfontfeatures{Scale=MatchLowercase}
  \defaultfontfeatures[\rmfamily]{Ligatures=TeX,Scale=1}
\fi
% Use upquote if available, for straight quotes in verbatim environments
\IfFileExists{upquote.sty}{\usepackage{upquote}}{}
\IfFileExists{microtype.sty}{% use microtype if available
  \usepackage[]{microtype}
  \UseMicrotypeSet[protrusion]{basicmath} % disable protrusion for tt fonts
}{}
\makeatletter
\@ifundefined{KOMAClassName}{% if non-KOMA class
  \IfFileExists{parskip.sty}{%
    \usepackage{parskip}
  }{% else
    \setlength{\parindent}{0pt}
    \setlength{\parskip}{6pt plus 2pt minus 1pt}}
}{% if KOMA class
  \KOMAoptions{parskip=half}}
\makeatother
\usepackage{xcolor}
\usepackage[margin=1in]{geometry}
\usepackage{color}
\usepackage{fancyvrb}
\newcommand{\VerbBar}{|}
\newcommand{\VERB}{\Verb[commandchars=\\\{\}]}
\DefineVerbatimEnvironment{Highlighting}{Verbatim}{commandchars=\\\{\}}
% Add ',fontsize=\small' for more characters per line
\usepackage{framed}
\definecolor{shadecolor}{RGB}{248,248,248}
\newenvironment{Shaded}{\begin{snugshade}}{\end{snugshade}}
\newcommand{\AlertTok}[1]{\textcolor[rgb]{0.94,0.16,0.16}{#1}}
\newcommand{\AnnotationTok}[1]{\textcolor[rgb]{0.56,0.35,0.01}{\textbf{\textit{#1}}}}
\newcommand{\AttributeTok}[1]{\textcolor[rgb]{0.77,0.63,0.00}{#1}}
\newcommand{\BaseNTok}[1]{\textcolor[rgb]{0.00,0.00,0.81}{#1}}
\newcommand{\BuiltInTok}[1]{#1}
\newcommand{\CharTok}[1]{\textcolor[rgb]{0.31,0.60,0.02}{#1}}
\newcommand{\CommentTok}[1]{\textcolor[rgb]{0.56,0.35,0.01}{\textit{#1}}}
\newcommand{\CommentVarTok}[1]{\textcolor[rgb]{0.56,0.35,0.01}{\textbf{\textit{#1}}}}
\newcommand{\ConstantTok}[1]{\textcolor[rgb]{0.00,0.00,0.00}{#1}}
\newcommand{\ControlFlowTok}[1]{\textcolor[rgb]{0.13,0.29,0.53}{\textbf{#1}}}
\newcommand{\DataTypeTok}[1]{\textcolor[rgb]{0.13,0.29,0.53}{#1}}
\newcommand{\DecValTok}[1]{\textcolor[rgb]{0.00,0.00,0.81}{#1}}
\newcommand{\DocumentationTok}[1]{\textcolor[rgb]{0.56,0.35,0.01}{\textbf{\textit{#1}}}}
\newcommand{\ErrorTok}[1]{\textcolor[rgb]{0.64,0.00,0.00}{\textbf{#1}}}
\newcommand{\ExtensionTok}[1]{#1}
\newcommand{\FloatTok}[1]{\textcolor[rgb]{0.00,0.00,0.81}{#1}}
\newcommand{\FunctionTok}[1]{\textcolor[rgb]{0.00,0.00,0.00}{#1}}
\newcommand{\ImportTok}[1]{#1}
\newcommand{\InformationTok}[1]{\textcolor[rgb]{0.56,0.35,0.01}{\textbf{\textit{#1}}}}
\newcommand{\KeywordTok}[1]{\textcolor[rgb]{0.13,0.29,0.53}{\textbf{#1}}}
\newcommand{\NormalTok}[1]{#1}
\newcommand{\OperatorTok}[1]{\textcolor[rgb]{0.81,0.36,0.00}{\textbf{#1}}}
\newcommand{\OtherTok}[1]{\textcolor[rgb]{0.56,0.35,0.01}{#1}}
\newcommand{\PreprocessorTok}[1]{\textcolor[rgb]{0.56,0.35,0.01}{\textit{#1}}}
\newcommand{\RegionMarkerTok}[1]{#1}
\newcommand{\SpecialCharTok}[1]{\textcolor[rgb]{0.00,0.00,0.00}{#1}}
\newcommand{\SpecialStringTok}[1]{\textcolor[rgb]{0.31,0.60,0.02}{#1}}
\newcommand{\StringTok}[1]{\textcolor[rgb]{0.31,0.60,0.02}{#1}}
\newcommand{\VariableTok}[1]{\textcolor[rgb]{0.00,0.00,0.00}{#1}}
\newcommand{\VerbatimStringTok}[1]{\textcolor[rgb]{0.31,0.60,0.02}{#1}}
\newcommand{\WarningTok}[1]{\textcolor[rgb]{0.56,0.35,0.01}{\textbf{\textit{#1}}}}
\usepackage{graphicx}
\makeatletter
\def\maxwidth{\ifdim\Gin@nat@width>\linewidth\linewidth\else\Gin@nat@width\fi}
\def\maxheight{\ifdim\Gin@nat@height>\textheight\textheight\else\Gin@nat@height\fi}
\makeatother
% Scale images if necessary, so that they will not overflow the page
% margins by default, and it is still possible to overwrite the defaults
% using explicit options in \includegraphics[width, height, ...]{}
\setkeys{Gin}{width=\maxwidth,height=\maxheight,keepaspectratio}
% Set default figure placement to htbp
\makeatletter
\def\fps@figure{htbp}
\makeatother
\setlength{\emergencystretch}{3em} % prevent overfull lines
\providecommand{\tightlist}{%
  \setlength{\itemsep}{0pt}\setlength{\parskip}{0pt}}
\setcounter{secnumdepth}{-\maxdimen} % remove section numbering
\ifLuaTeX
  \usepackage{selnolig}  % disable illegal ligatures
\fi
\IfFileExists{bookmark.sty}{\usepackage{bookmark}}{\usepackage{hyperref}}
\IfFileExists{xurl.sty}{\usepackage{xurl}}{} % add URL line breaks if available
\urlstyle{same} % disable monospaced font for URLs
\hypersetup{
  pdftitle={INFS 692 - Final Project Model 1},
  pdfauthor={Agata},
  hidelinks,
  pdfcreator={LaTeX via pandoc}}

\title{INFS 692 - Final Project Model 1}
\author{Agata}
\date{2022-12-14}

\begin{document}
\maketitle

\hypertarget{helper-packages}{%
\subsection{Helper Packages}\label{helper-packages}}

These are all the packages necessary for running the Model 1 code.

\begin{Shaded}
\begin{Highlighting}[]
\FunctionTok{library}\NormalTok{(readr)}
\FunctionTok{library}\NormalTok{(plyr)}
\FunctionTok{library}\NormalTok{(dplyr)}
\end{Highlighting}
\end{Shaded}

\begin{verbatim}
## 
## Attaching package: 'dplyr'
\end{verbatim}

\begin{verbatim}
## The following objects are masked from 'package:plyr':
## 
##     arrange, count, desc, failwith, id, mutate, rename, summarise,
##     summarize
\end{verbatim}

\begin{verbatim}
## The following objects are masked from 'package:stats':
## 
##     filter, lag
\end{verbatim}

\begin{verbatim}
## The following objects are masked from 'package:base':
## 
##     intersect, setdiff, setequal, union
\end{verbatim}

\begin{Shaded}
\begin{Highlighting}[]
\FunctionTok{library}\NormalTok{(ggplot2)}
\FunctionTok{library}\NormalTok{(ggpubr)}
\end{Highlighting}
\end{Shaded}

\begin{verbatim}
## 
## Attaching package: 'ggpubr'
\end{verbatim}

\begin{verbatim}
## The following object is masked from 'package:plyr':
## 
##     mutate
\end{verbatim}

\begin{Shaded}
\begin{Highlighting}[]
\FunctionTok{library}\NormalTok{(gridExtra)}
\end{Highlighting}
\end{Shaded}

\begin{verbatim}
## 
## Attaching package: 'gridExtra'
\end{verbatim}

\begin{verbatim}
## The following object is masked from 'package:dplyr':
## 
##     combine
\end{verbatim}

\begin{Shaded}
\begin{Highlighting}[]
\FunctionTok{library}\NormalTok{(COUNT)}
\end{Highlighting}
\end{Shaded}

\begin{verbatim}
## Loading required package: msme
\end{verbatim}

\begin{verbatim}
## Loading required package: MASS
\end{verbatim}

\begin{verbatim}
## 
## Attaching package: 'MASS'
\end{verbatim}

\begin{verbatim}
## The following object is masked from 'package:dplyr':
## 
##     select
\end{verbatim}

\begin{verbatim}
## Loading required package: lattice
\end{verbatim}

\begin{verbatim}
## Loading required package: sandwich
\end{verbatim}

\begin{Shaded}
\begin{Highlighting}[]
\FunctionTok{library}\NormalTok{(caret)}
\FunctionTok{library}\NormalTok{(rstatix)}
\end{Highlighting}
\end{Shaded}

\begin{verbatim}
## 
## Attaching package: 'rstatix'
\end{verbatim}

\begin{verbatim}
## The following object is masked from 'package:MASS':
## 
##     select
\end{verbatim}

\begin{verbatim}
## The following objects are masked from 'package:plyr':
## 
##     desc, mutate
\end{verbatim}

\begin{verbatim}
## The following object is masked from 'package:stats':
## 
##     filter
\end{verbatim}

\begin{Shaded}
\begin{Highlighting}[]
\FunctionTok{library}\NormalTok{(modeldata)}
\FunctionTok{library}\NormalTok{(rsample)    }\CommentTok{\# for creating validation splits}
\FunctionTok{library}\NormalTok{(recipes)    }\CommentTok{\# for feature engineering}
\end{Highlighting}
\end{Shaded}

\begin{verbatim}
## 
## Attaching package: 'recipes'
\end{verbatim}

\begin{verbatim}
## The following object is masked from 'package:stats':
## 
##     step
\end{verbatim}

\begin{Shaded}
\begin{Highlighting}[]
\FunctionTok{library}\NormalTok{(purrr)      }\CommentTok{\#for mapping}
\end{Highlighting}
\end{Shaded}

\begin{verbatim}
## 
## Attaching package: 'purrr'
\end{verbatim}

\begin{verbatim}
## The following object is masked from 'package:caret':
## 
##     lift
\end{verbatim}

\begin{verbatim}
## The following object is masked from 'package:plyr':
## 
##     compact
\end{verbatim}

\begin{Shaded}
\begin{Highlighting}[]
\FunctionTok{library}\NormalTok{(tidyverse)  }\CommentTok{\# for filtering }
\end{Highlighting}
\end{Shaded}

\begin{verbatim}
## -- Attaching packages --------------------------------------- tidyverse 1.3.2 --
\end{verbatim}

\begin{verbatim}
## v tibble  3.1.8     v stringr 1.5.0
## v tidyr   1.2.1     v forcats 0.5.2
## -- Conflicts ------------------------------------------ tidyverse_conflicts() --
## x dplyr::arrange()     masks plyr::arrange()
## x gridExtra::combine() masks dplyr::combine()
## x purrr::compact()     masks plyr::compact()
## x dplyr::count()       masks plyr::count()
## x dplyr::failwith()    masks plyr::failwith()
## x rstatix::filter()    masks dplyr::filter(), stats::filter()
## x stringr::fixed()     masks recipes::fixed()
## x dplyr::id()          masks plyr::id()
## x dplyr::lag()         masks stats::lag()
## x purrr::lift()        masks caret::lift()
## x rstatix::mutate()    masks ggpubr::mutate(), dplyr::mutate(), plyr::mutate()
## x dplyr::rename()      masks plyr::rename()
## x rstatix::select()    masks MASS::select(), dplyr::select()
## x dplyr::summarise()   masks plyr::summarise()
## x dplyr::summarize()   masks plyr::summarize()
\end{verbatim}

\begin{Shaded}
\begin{Highlighting}[]
\FunctionTok{library}\NormalTok{(ROCR)      }\CommentTok{\# ROC Curves}
\FunctionTok{library}\NormalTok{(pROC)      }\CommentTok{\# ROC Curves}
\end{Highlighting}
\end{Shaded}

\begin{verbatim}
## Type 'citation("pROC")' for a citation.
## 
## Attaching package: 'pROC'
## 
## The following objects are masked from 'package:stats':
## 
##     cov, smooth, var
\end{verbatim}

\begin{Shaded}
\begin{Highlighting}[]
\FunctionTok{library}\NormalTok{(rpart)      }\CommentTok{\# decision tree application}
\FunctionTok{library}\NormalTok{(rpart.plot)  }\CommentTok{\# plotting decision trees}
\FunctionTok{library}\NormalTok{(vip)         }\CommentTok{\# for feature importance}
\end{Highlighting}
\end{Shaded}

\begin{verbatim}
## 
## Attaching package: 'vip'
## 
## The following object is masked from 'package:utils':
## 
##     vi
\end{verbatim}

\begin{Shaded}
\begin{Highlighting}[]
\FunctionTok{library}\NormalTok{(pdp) }
\end{Highlighting}
\end{Shaded}

\begin{verbatim}
## 
## Attaching package: 'pdp'
## 
## The following object is masked from 'package:purrr':
## 
##     partial
\end{verbatim}

\hypertarget{loading-data}{%
\subsection{Loading Data}\label{loading-data}}

Before pre=processing the data, we need to load it into a data1 variable
for easier manipulation. The str() function helps identify which
variables in the dataset are categorical, factors, etc.

\begin{Shaded}
\begin{Highlighting}[]
\NormalTok{data1 }\OtherTok{\textless{}{-}} \FunctionTok{read.csv}\NormalTok{(}\StringTok{"radiomics\_completedata.csv"}\NormalTok{, }\AttributeTok{sep =} \StringTok{","}\NormalTok{)}
\end{Highlighting}
\end{Shaded}

\hypertarget{preprocessing-data}{%
\subsection{Preprocessing Data}\label{preprocessing-data}}

First, we must check the data for null and missing values.

\begin{Shaded}
\begin{Highlighting}[]
\CommentTok{\#Check for null and missing values}

\FunctionTok{which}\NormalTok{(}\FunctionTok{is.null}\NormalTok{(data1))}
\end{Highlighting}
\end{Shaded}

\begin{verbatim}
## integer(0)
\end{verbatim}

\begin{Shaded}
\begin{Highlighting}[]
\FunctionTok{which}\NormalTok{(}\FunctionTok{is.na}\NormalTok{(data1))}
\end{Highlighting}
\end{Shaded}

\begin{verbatim}
## integer(0)
\end{verbatim}

In this case, we have neither missing nor null values. We can proceed to
splitting the data.

We want to split the data so that it doesn't include any categorical
variables, or the Failure column.

\begin{Shaded}
\begin{Highlighting}[]
\CommentTok{\#Data split}

\NormalTok{sub1 }\OtherTok{\textless{}{-}} \FunctionTok{subset}\NormalTok{(data1, }\AttributeTok{select=} \SpecialCharTok{{-}}\FunctionTok{c}\NormalTok{(Institution, Failure))}
\end{Highlighting}
\end{Shaded}

Next, we must check if the data has a normal distribution. We can do
this using two methods: using a histogram and determining visually if it
has a bell curve (meaning data is normalized), or using the Shapiro test
in which if the p-value is \textless{} than 0.5, that means that the
data is not normally distributed.

\begin{Shaded}
\begin{Highlighting}[]
\CommentTok{\#Check for normality}

\FunctionTok{hist}\NormalTok{(sub1}\SpecialCharTok{$}\NormalTok{GLNU\_align.H.PET)}
\end{Highlighting}
\end{Shaded}

\includegraphics{INFS-692---Final-Project-Model-1_files/figure-latex/unnamed-chunk-5-1.pdf}

\begin{Shaded}
\begin{Highlighting}[]
\NormalTok{sub1shapiro }\OtherTok{\textless{}{-}} \FunctionTok{shapiro.test}\NormalTok{(sub1}\SpecialCharTok{$}\NormalTok{GLNU\_align.H.PET)}
\NormalTok{sub1shapiro}
\end{Highlighting}
\end{Shaded}

\begin{verbatim}
## 
##  Shapiro-Wilk normality test
## 
## data:  sub1$GLNU_align.H.PET
## W = 0.76271, p-value < 2.2e-16
\end{verbatim}

In this case, the histogram doesn't show us a bell curve, and the
p-value from the Shapiro test is \textless{} 0.5, meaning data is not
distributed normally.

We must perform data normalization using the scale() function. Once
done, to check if data is normalized, we can use the summary() function
to see if the mean is = 0 and use the sd() function to see if the
standard deviation is = 1.

\begin{Shaded}
\begin{Highlighting}[]
\CommentTok{\#Normalize Data}

\NormalTok{scale\_data }\OtherTok{\textless{}{-}}  \FunctionTok{as.data.frame}\NormalTok{(}\FunctionTok{scale}\NormalTok{(sub1, }\AttributeTok{center =} \ConstantTok{TRUE}\NormalTok{, }\AttributeTok{scale =} \ConstantTok{TRUE}\NormalTok{))}

\FunctionTok{summary}\NormalTok{(scale\_data}\SpecialCharTok{$}\NormalTok{GLNU\_align.H.PET)}
\end{Highlighting}
\end{Shaded}

\begin{verbatim}
##    Min. 1st Qu.  Median    Mean 3rd Qu.    Max. 
## -0.9982 -0.6721 -0.1783  0.0000  0.1947  5.3894
\end{verbatim}

\begin{Shaded}
\begin{Highlighting}[]
\FunctionTok{sd}\NormalTok{(scale\_data}\SpecialCharTok{$}\NormalTok{GLNU\_align.H.PET)}
\end{Highlighting}
\end{Shaded}

\begin{verbatim}
## [1] 1
\end{verbatim}

Now the data has a mean of 0 and a standard deviation of 1, meaning the
data is normalized.

We then check correlation of the full dataset without the categorical
variables:

\begin{Shaded}
\begin{Highlighting}[]
\NormalTok{cor1 }\OtherTok{\textless{}{-}} \FunctionTok{cor}\NormalTok{(}\FunctionTok{select}\NormalTok{(scale\_data, }\SpecialCharTok{{-}}\FunctionTok{c}\NormalTok{(Failure.binary)))}
\CommentTok{\#cor1 this has been commented out or else there would be 700 pages in the pdf}
\end{Highlighting}
\end{Shaded}

\hypertarget{dataset-training-and-testing-split}{%
\subsubsection{Dataset Training and Testing
Split}\label{dataset-training-and-testing-split}}

With preprocessing done, we can split the training and testing dataset.
For this, we begin by factoring the Failure.binary column, and
transforming the levels so that they represent Failure and Success. This
is important for KNN and Decision Trees.

For memory purposes and faster processing, we split the dataframe to
take only a ``sample'' of the full dataset, or just the 50 first
columns. The training is split at 80\%, using Failure.binary as the
output.

\begin{Shaded}
\begin{Highlighting}[]
\NormalTok{scale\_data}\SpecialCharTok{$}\NormalTok{Failure.binary }\OtherTok{\textless{}{-}} \FunctionTok{factor}\NormalTok{(data1}\SpecialCharTok{$}\NormalTok{Failure.binary)}


\CommentTok{\#data1 \textless{}{-} select(scale\_data, {-}c("Institution", "Failure"))}

\NormalTok{data1 }\OtherTok{\textless{}{-}} \FunctionTok{select}\NormalTok{(scale\_data, }\DecValTok{1}\SpecialCharTok{:}\DecValTok{50}\NormalTok{)}

\FunctionTok{levels}\NormalTok{(data1}\SpecialCharTok{$}\NormalTok{Failure.binary) }\OtherTok{\textless{}{-}} \FunctionTok{c}\NormalTok{(}\StringTok{"Failure"}\NormalTok{, }\StringTok{"Success"}\NormalTok{)}


\NormalTok{df }\OtherTok{\textless{}{-}}\NormalTok{ data1}

\FunctionTok{str}\NormalTok{(df)}
\end{Highlighting}
\end{Shaded}

\begin{verbatim}
## 'data.frame':    197 obs. of  50 variables:
##  $ Failure.binary             : Factor w/ 2 levels "Failure","Success": 1 2 1 2 1 2 1 1 2 2 ...
##  $ Entropy_cooc.W.ADC         : num  0.5529 -0.0649 0.4599 1.1432 0.345 ...
##  $ GLNU_align.H.PET           : num  -0.5706 -0.789 -0.0602 2.6747 -0.0674 ...
##  $ Min_hist.PET               : num  -0.454 0.5 -1.15 -0.445 -0.989 ...
##  $ Max_hist.PET               : num  -0.436 0.149 -1.177 -0.152 -1.106 ...
##  $ Mean_hist.PET              : num  -0.42 0.315 -1.136 -0.349 -1.116 ...
##  $ Variance_hist.PET          : num  -0.263 0.395 -0.896 -0.28 -0.934 ...
##  $ Standard_Deviation_hist.PET: num  -0.236 0.297 -1.129 -0.253 -1.24 ...
##  $ Skewness_hist.PET          : num  -0.323 -0.177 -0.959 -0.116 0.958 ...
##  $ Kurtosis_hist.PET          : num  -0.273 -0.266 -0.472 0.12 0.907 ...
##  $ Energy_hist.PET            : num  0.0502 0.0919 0.0474 -0.0124 0.1533 ...
##  $ Entropy_hist.PET           : num  -0.38 -0.747 -0.37 -0.157 -0.853 ...
##  $ AUC_hist.PET               : num  -0.568 -0.563 -0.581 -0.407 -0.408 ...
##  $ H_suv.PET                  : num  -0.121 0.95 -1.072 -0.393 -1.211 ...
##  $ Volume.PET                 : num  -0.7713 -0.8698 -0.4849 0.0587 -0.4229 ...
##  $ X3D_surface.PET            : num  -0.52 -0.431 -0.155 0.244 -0.45 ...
##  $ ratio_3ds_vol.PET          : num  -0.228 0.422 -0.248 -0.701 0.409 ...
##  $ ratio_3ds_vol_norm.PET     : num  -0.37675 0.00118 -0.11356 -0.06927 -0.00444 ...
##  $ irregularity.PET           : num  -0.404 -0.259 -0.501 -0.779 -0.396 ...
##  $ tumor_length.PET           : num  -0.499 -0.625 -0.314 0.368 -0.691 ...
##  $ Compactness_v1.PET         : num  -0.072 -0.0845 -0.0816 -0.0828 -0.0844 ...
##  $ Compactness_v2.PET         : num  -0.425 -0.427 -0.426 -0.426 -0.427 ...
##  $ Spherical_disproportion.PET: num  -0.37675 0.00118 -0.11356 -0.06927 -0.00444 ...
##  $ Sphericity.PET             : num  -0.443 -0.505 -0.49 -0.496 -0.504 ...
##  $ Asphericity.PET            : num  -0.3646 0.0201 -0.0967 -0.0517 0.0143 ...
##  $ Center_of_mass.PET         : num  -0.0305 -0.3264 -0.5841 0.0433 -0.4082 ...
##  $ Max_3D_diam.PET            : num  -0.6641 -0.7524 -0.5337 -0.0528 -0.7991 ...
##  $ Major_axis_length.PET      : num  -0.7799 -0.7671 -0.4524 -0.0649 -0.7462 ...
##  $ Minor_axis_length.PET      : num  -0.81 -0.749 -0.616 0.43 -0.899 ...
##  $ Least_axis_length.PET      : num  -0.553 -0.74 -0.43 0.74 -0.728 ...
##  $ Elongation.PET             : num  -0.377 -0.3 -0.683 -0.111 -0.601 ...
##  $ Flatness.PET               : num  0.0389 -0.3472 -0.4444 0.3031 -0.3724 ...
##  $ Max_cooc.L.PET             : num  0.0191 0.1307 0.0195 0.0526 0.1083 ...
##  $ Average_cooc.L.PET         : num  -0.3868 -0.4758 0.0139 -0.8511 -1.0757 ...
##  $ Variance_cooc.L.PET        : num  -0.1075 0.0906 -0.0764 -1.0807 -0.7069 ...
##  $ Entropy_cooc.L.PET         : num  -0.498 -0.586 -0.456 -0.598 -0.688 ...
##  $ DAVE_cooc.L.PET            : num  -0.3221 0.0172 -0.2548 -1.0184 -0.5794 ...
##  $ DVAR_cooc.L.PET            : num  -0.438 0.284 -0.42 -1.081 -0.515 ...
##  $ DENT_cooc.L.PET            : num  -0.489 -0.392 -0.485 -0.774 -0.58 ...
##  $ SAVE_cooc.L.PET            : num  -0.3871 -0.4761 0.0138 -0.8516 -1.0763 ...
##  $ SVAR_cooc.L.PET            : num  -0.0267 -0.0503 0.0164 -1.0376 -0.7682 ...
##  $ SENT_cooc.L.PET            : num  -0.437 -0.452 -0.416 -0.592 -0.614 ...
##  $ ASM_cooc.L.PET             : num  0.0857 0.0965 0.0819 0.0996 0.1113 ...
##  $ Contrast_cooc.L.PET        : num  -0.221 0.302 -0.214 -1.004 -0.515 ...
##  $ Dissimilarity_cooc.L.PET   : num  -0.3221 0.0172 -0.2548 -1.0184 -0.5794 ...
##  $ Inv_diff_cooc.L.PET        : num  -0.5668 -0.6568 -0.673 0.0153 -0.3554 ...
##  $ Inv_diff_norm_cooc.L.PET   : num  -0.576 -0.626 -0.591 -0.458 -0.533 ...
##  $ IDM_cooc.L.PET             : num  -0.53 -0.577 -0.66 0.158 -0.277 ...
##  $ IDM_norm_cooc.L.PET        : num  -0.567 -0.605 -0.57 -0.506 -0.544 ...
##  $ Inv_var_cooc.L.PET         : num  -0.533 -0.581 -0.618 0.213 -0.239 ...
\end{verbatim}

\begin{Shaded}
\begin{Highlighting}[]
\CommentTok{\# Create training (80\%) and test (20\%) sets for the }
\FunctionTok{set.seed}\NormalTok{(}\DecValTok{123}\NormalTok{)  }\CommentTok{\# for reproducibility}
\NormalTok{churn\_split }\OtherTok{\textless{}{-}} \FunctionTok{initial\_split}\NormalTok{(df, }\AttributeTok{prop =} \FloatTok{0.8}\NormalTok{, }\AttributeTok{strata =} \StringTok{"Failure.binary"}\NormalTok{)}
\NormalTok{churn\_train }\OtherTok{\textless{}{-}} \FunctionTok{training}\NormalTok{(churn\_split)}
\NormalTok{churn\_test  }\OtherTok{\textless{}{-}} \FunctionTok{testing}\NormalTok{(churn\_split)}
\end{Highlighting}
\end{Shaded}

\hypertarget{model-1-linear-regression}{%
\subsubsection{Model 1: Linear
Regression}\label{model-1-linear-regression}}

The first model we will look at is Linear Regression.

First, we train 3 different models.

\begin{Shaded}
\begin{Highlighting}[]
\CommentTok{\#Model training}
 \FunctionTok{set.seed}\NormalTok{(}\DecValTok{123}\NormalTok{)}
\NormalTok{ cv\_model1 }\OtherTok{\textless{}{-}} \FunctionTok{train}\NormalTok{(}
\NormalTok{   Failure.binary }\SpecialCharTok{\textasciitilde{}}\NormalTok{ H\_suv.PET, }
   \AttributeTok{data =}\NormalTok{ churn\_train, }
   \AttributeTok{method =} \StringTok{"glm"}\NormalTok{,}
   \AttributeTok{family =} \StringTok{"binomial"}\NormalTok{,}
   \AttributeTok{trControl =} \FunctionTok{trainControl}\NormalTok{(}\AttributeTok{method =} \StringTok{"cv"}\NormalTok{, }\AttributeTok{number =} \DecValTok{5}\NormalTok{)}
\NormalTok{   )}

\FunctionTok{set.seed}\NormalTok{(}\DecValTok{123}\NormalTok{)}
\NormalTok{cv\_model2 }\OtherTok{\textless{}{-}} \FunctionTok{train}\NormalTok{(}
\NormalTok{   Failure.binary }\SpecialCharTok{\textasciitilde{}}\NormalTok{ Entropy\_cooc.W.ADC }\SpecialCharTok{+}\NormalTok{ GLNU\_align.H.PET,}
   \AttributeTok{data =}\NormalTok{ churn\_train,}
   \AttributeTok{method =} \StringTok{"glm"}\NormalTok{,}
   \AttributeTok{family =} \StringTok{"binomial"}\NormalTok{,}
   \AttributeTok{trControl =} \FunctionTok{trainControl}\NormalTok{(}\AttributeTok{method =} \StringTok{"cv"}\NormalTok{, }\AttributeTok{number =} \DecValTok{5}\NormalTok{)}

\NormalTok{ )}

 \FunctionTok{set.seed}\NormalTok{(}\DecValTok{123}\NormalTok{)}
\NormalTok{ cv\_model3 }\OtherTok{\textless{}{-}} \FunctionTok{train}\NormalTok{(}
\NormalTok{   Failure.binary }\SpecialCharTok{\textasciitilde{}}\NormalTok{ .,  }\CommentTok{\#overall datasets}
   \AttributeTok{data =}\NormalTok{ churn\_train,}
   \AttributeTok{method =} \StringTok{"glm"}\NormalTok{,}
   \AttributeTok{family =} \StringTok{"binomial"}\NormalTok{,}
   \AttributeTok{trControl =} \FunctionTok{trainControl}\NormalTok{(}\AttributeTok{method =} \StringTok{"cv"}\NormalTok{, }\AttributeTok{number =} \DecValTok{5}\NormalTok{)}
\NormalTok{ )}
\end{Highlighting}
\end{Shaded}

\begin{verbatim}
## Warning: glm.fit: fitted probabilities numerically 0 or 1 occurred
\end{verbatim}

\begin{verbatim}
## Warning in predict.lm(object, newdata, se.fit, scale = 1, type = if (type == :
## prediction from a rank-deficient fit may be misleading
\end{verbatim}

\begin{verbatim}
## Warning: glm.fit: algorithm did not converge
\end{verbatim}

\begin{verbatim}
## Warning: glm.fit: fitted probabilities numerically 0 or 1 occurred
\end{verbatim}

\begin{verbatim}
## Warning in predict.lm(object, newdata, se.fit, scale = 1, type = if (type == :
## prediction from a rank-deficient fit may be misleading
\end{verbatim}

\begin{verbatim}
## Warning: glm.fit: fitted probabilities numerically 0 or 1 occurred
\end{verbatim}

\begin{verbatim}
## Warning in predict.lm(object, newdata, se.fit, scale = 1, type = if (type == :
## prediction from a rank-deficient fit may be misleading
\end{verbatim}

\begin{verbatim}
## Warning: glm.fit: fitted probabilities numerically 0 or 1 occurred
\end{verbatim}

\begin{verbatim}
## Warning in predict.lm(object, newdata, se.fit, scale = 1, type = if (type == :
## prediction from a rank-deficient fit may be misleading
\end{verbatim}

\begin{verbatim}
## Warning: glm.fit: algorithm did not converge
\end{verbatim}

\begin{verbatim}
## Warning: glm.fit: fitted probabilities numerically 0 or 1 occurred
\end{verbatim}

\begin{verbatim}
## Warning in predict.lm(object, newdata, se.fit, scale = 1, type = if (type == :
## prediction from a rank-deficient fit may be misleading
\end{verbatim}

\begin{verbatim}
## Warning: glm.fit: fitted probabilities numerically 0 or 1 occurred
\end{verbatim}

Then we extract the sample performance measures:

\begin{Shaded}
\begin{Highlighting}[]
\CommentTok{\# extract out of sample performance measures}
\FunctionTok{summary}\NormalTok{(}
  \FunctionTok{resamples}\NormalTok{(}
    \FunctionTok{list}\NormalTok{(}
      \AttributeTok{model1 =}\NormalTok{ cv\_model1,}
      \AttributeTok{model2 =}\NormalTok{ cv\_model2,}
      \AttributeTok{model3 =}\NormalTok{ cv\_model3}
\NormalTok{    )}
\NormalTok{  )}
\NormalTok{)}\SpecialCharTok{$}\NormalTok{statistics}\SpecialCharTok{$}\NormalTok{Accuracy}
\end{Highlighting}
\end{Shaded}

\begin{verbatim}
##             Min.   1st Qu.    Median      Mean   3rd Qu.      Max. NA's
## model1 0.6451613 0.6562500 0.6562500 0.6625000 0.6774194 0.6774194    0
## model2 0.7741935 0.8064516 0.8437500 0.8530242 0.9032258 0.9375000    0
## model3 0.6250000 0.6875000 0.7096774 0.7141129 0.7741935 0.7741935    0
\end{verbatim}

As seen above, model 2 has the best results. As a whole, model 1 is the
weakest (Final.binary on its own).

Next, we create the prediction classes for each model and their
confusion matrices.

Because we changed the levels for Faliure.binary to ``Failure'' and
``Success'' we need to change the reference parameters to that.

\begin{Shaded}
\begin{Highlighting}[]
\CommentTok{\# predict class}
\NormalTok{pred\_class\_1 }\OtherTok{\textless{}{-}} \FunctionTok{predict}\NormalTok{(cv\_model1, churn\_train)}


\CommentTok{\#balanced accuracy is most important}

\CommentTok{\# create confusion matrix}
\FunctionTok{confusionMatrix}\NormalTok{(}
  \AttributeTok{data =} \FunctionTok{relevel}\NormalTok{(pred\_class\_1, }\AttributeTok{ref =} \StringTok{"Success"}\NormalTok{), }
  \AttributeTok{reference =} \FunctionTok{relevel}\NormalTok{(churn\_train}\SpecialCharTok{$}\NormalTok{Failure.binary, }\AttributeTok{ref =} \StringTok{"Success"}\NormalTok{)}
\NormalTok{)}
\end{Highlighting}
\end{Shaded}

\begin{verbatim}
## Confusion Matrix and Statistics
## 
##           Reference
## Prediction Success Failure
##    Success       0       0
##    Failure      53     104
##                                           
##                Accuracy : 0.6624          
##                  95% CI : (0.5827, 0.7359)
##     No Information Rate : 0.6624          
##     P-Value [Acc > NIR] : 0.5372          
##                                           
##                   Kappa : 0               
##                                           
##  Mcnemar's Test P-Value : 9.148e-13       
##                                           
##             Sensitivity : 0.0000          
##             Specificity : 1.0000          
##          Pos Pred Value :    NaN          
##          Neg Pred Value : 0.6624          
##              Prevalence : 0.3376          
##          Detection Rate : 0.0000          
##    Detection Prevalence : 0.0000          
##       Balanced Accuracy : 0.5000          
##                                           
##        'Positive' Class : Success         
## 
\end{verbatim}

\begin{Shaded}
\begin{Highlighting}[]
\NormalTok{pred\_class\_2 }\OtherTok{\textless{}{-}} \FunctionTok{predict}\NormalTok{(cv\_model2, churn\_train)}

\CommentTok{\# create confusion matrix}
\FunctionTok{confusionMatrix}\NormalTok{(}
  \AttributeTok{data =} \FunctionTok{relevel}\NormalTok{(pred\_class\_2, }\AttributeTok{ref =} \StringTok{"Success"}\NormalTok{), }
  \AttributeTok{reference =} \FunctionTok{relevel}\NormalTok{(churn\_train}\SpecialCharTok{$}\NormalTok{Failure.binary, }\AttributeTok{ref =} \StringTok{"Success"}\NormalTok{)}
\NormalTok{)}
\end{Highlighting}
\end{Shaded}

\begin{verbatim}
## Confusion Matrix and Statistics
## 
##           Reference
## Prediction Success Failure
##    Success      41      10
##    Failure      12      94
##                                           
##                Accuracy : 0.8599          
##                  95% CI : (0.7956, 0.9101)
##     No Information Rate : 0.6624          
##     P-Value [Acc > NIR] : 1.688e-08       
##                                           
##                   Kappa : 0.6838          
##                                           
##  Mcnemar's Test P-Value : 0.8312          
##                                           
##             Sensitivity : 0.7736          
##             Specificity : 0.9038          
##          Pos Pred Value : 0.8039          
##          Neg Pred Value : 0.8868          
##              Prevalence : 0.3376          
##          Detection Rate : 0.2611          
##    Detection Prevalence : 0.3248          
##       Balanced Accuracy : 0.8387          
##                                           
##        'Positive' Class : Success         
## 
\end{verbatim}

\begin{Shaded}
\begin{Highlighting}[]
\NormalTok{pred\_class\_3 }\OtherTok{\textless{}{-}} \FunctionTok{predict}\NormalTok{(cv\_model3, churn\_train)}
\end{Highlighting}
\end{Shaded}

\begin{verbatim}
## Warning in predict.lm(object, newdata, se.fit, scale = 1, type = if (type == :
## prediction from a rank-deficient fit may be misleading
\end{verbatim}

\begin{Shaded}
\begin{Highlighting}[]
\CommentTok{\# create confusion matrix}
\FunctionTok{confusionMatrix}\NormalTok{(}
  \AttributeTok{data =} \FunctionTok{relevel}\NormalTok{(pred\_class\_3, }\AttributeTok{ref =} \StringTok{"Success"}\NormalTok{), }
  \AttributeTok{reference =} \FunctionTok{relevel}\NormalTok{(churn\_train}\SpecialCharTok{$}\NormalTok{Failure.binary, }\AttributeTok{ref =} \StringTok{"Success"}\NormalTok{)}
\NormalTok{)}
\end{Highlighting}
\end{Shaded}

\begin{verbatim}
## Confusion Matrix and Statistics
## 
##           Reference
## Prediction Success Failure
##    Success      40       3
##    Failure      13     101
##                                           
##                Accuracy : 0.8981          
##                  95% CI : (0.8398, 0.9406)
##     No Information Rate : 0.6624          
##     P-Value [Acc > NIR] : 6.463e-12       
##                                           
##                   Kappa : 0.7611          
##                                           
##  Mcnemar's Test P-Value : 0.02445         
##                                           
##             Sensitivity : 0.7547          
##             Specificity : 0.9712          
##          Pos Pred Value : 0.9302          
##          Neg Pred Value : 0.8860          
##              Prevalence : 0.3376          
##          Detection Rate : 0.2548          
##    Detection Prevalence : 0.2739          
##       Balanced Accuracy : 0.8629          
##                                           
##        'Positive' Class : Success         
## 
\end{verbatim}

\begin{Shaded}
\begin{Highlighting}[]
\CommentTok{\# Compute predicted probabilities on training data}
\NormalTok{m1\_prob }\OtherTok{\textless{}{-}} \FunctionTok{predict}\NormalTok{(cv\_model1, churn\_train, }\AttributeTok{type =} \StringTok{"prob"}\NormalTok{)}\SpecialCharTok{$}\NormalTok{Success}
\NormalTok{m2\_prob }\OtherTok{\textless{}{-}} \FunctionTok{predict}\NormalTok{(cv\_model2, churn\_train, }\AttributeTok{type =} \StringTok{"prob"}\NormalTok{)}\SpecialCharTok{$}\NormalTok{Success}
\NormalTok{m3\_prob }\OtherTok{\textless{}{-}} \FunctionTok{predict}\NormalTok{(cv\_model3, churn\_train, }\AttributeTok{type =} \StringTok{"prob"}\NormalTok{)}\SpecialCharTok{$}\NormalTok{Success}
\end{Highlighting}
\end{Shaded}

\begin{verbatim}
## Warning in predict.lm(object, newdata, se.fit, scale = 1, type = if (type == :
## prediction from a rank-deficient fit may be misleading
\end{verbatim}

\begin{Shaded}
\begin{Highlighting}[]
\CommentTok{\# Compute AUC metrics for cv\_model1,2 and 3 }
\NormalTok{perf1 }\OtherTok{\textless{}{-}} \FunctionTok{prediction}\NormalTok{(m1\_prob, churn\_train}\SpecialCharTok{$}\NormalTok{Failure.binary) }\SpecialCharTok{\%\textgreater{}\%}
  \FunctionTok{performance}\NormalTok{(}\AttributeTok{measure =} \StringTok{"tpr"}\NormalTok{, }\AttributeTok{x.measure =} \StringTok{"fpr"}\NormalTok{)}
\NormalTok{perf2 }\OtherTok{\textless{}{-}} \FunctionTok{prediction}\NormalTok{(m2\_prob, churn\_train}\SpecialCharTok{$}\NormalTok{Failure.binary) }\SpecialCharTok{\%\textgreater{}\%}
  \FunctionTok{performance}\NormalTok{(}\AttributeTok{measure =} \StringTok{"tpr"}\NormalTok{, }\AttributeTok{x.measure =} \StringTok{"fpr"}\NormalTok{)}
\NormalTok{perf3 }\OtherTok{\textless{}{-}} \FunctionTok{prediction}\NormalTok{(m3\_prob, churn\_train}\SpecialCharTok{$}\NormalTok{Failure.binary) }\SpecialCharTok{\%\textgreater{}\%}
  \FunctionTok{performance}\NormalTok{(}\AttributeTok{measure =} \StringTok{"tpr"}\NormalTok{, }\AttributeTok{x.measure =} \StringTok{"fpr"}\NormalTok{)}

\CommentTok{\# Plot ROC curves for cv\_model1,2 and 3 }
\FunctionTok{plot}\NormalTok{(perf1, }\AttributeTok{col =} \StringTok{"black"}\NormalTok{, }\AttributeTok{lty =} \DecValTok{2}\NormalTok{)}
\FunctionTok{plot}\NormalTok{(perf2,  }\AttributeTok{add =} \ConstantTok{TRUE}\NormalTok{, }\AttributeTok{col =} \StringTok{"red"}\NormalTok{, }\AttributeTok{lty =} \DecValTok{2}\NormalTok{)}
\FunctionTok{plot}\NormalTok{(perf3, }\AttributeTok{add =} \ConstantTok{TRUE}\NormalTok{, }\AttributeTok{col =} \StringTok{"blue"}\NormalTok{)}
\FunctionTok{legend}\NormalTok{(}\FloatTok{0.7}\NormalTok{, }\FloatTok{0.3}\NormalTok{, }\AttributeTok{legend =} \FunctionTok{c}\NormalTok{(}\StringTok{"cv\_model1"}\NormalTok{, }\StringTok{"cv\_model2"}\NormalTok{, }\StringTok{"cv\_model3"}\NormalTok{),}
       \AttributeTok{col =} \FunctionTok{c}\NormalTok{(}\StringTok{"black"}\NormalTok{,}\StringTok{"red"}\NormalTok{, }\StringTok{"blue"}\NormalTok{), }\AttributeTok{lty =} \DecValTok{3}\SpecialCharTok{:}\DecValTok{1}\NormalTok{, }\AttributeTok{cex =} \FloatTok{0.6}\NormalTok{)}
\end{Highlighting}
\end{Shaded}

\includegraphics{INFS-692---Final-Project-Model-1_files/figure-latex/unnamed-chunk-12-1.pdf}

\begin{Shaded}
\begin{Highlighting}[]
\CommentTok{\# ROC plot for training data}
\FunctionTok{roc}\NormalTok{(churn\_train}\SpecialCharTok{$}\NormalTok{Failure.binary }\SpecialCharTok{\textasciitilde{}}\NormalTok{ m1\_prob, }\AttributeTok{plot=}\ConstantTok{TRUE}\NormalTok{, }\AttributeTok{legacy.axes=}\ConstantTok{FALSE}\NormalTok{, }
    \AttributeTok{percent=}\ConstantTok{TRUE}\NormalTok{, }\AttributeTok{col=}\StringTok{"black"}\NormalTok{, }\AttributeTok{lwd=}\DecValTok{2}\NormalTok{, }\AttributeTok{print.auc=}\ConstantTok{TRUE}\NormalTok{)}
\end{Highlighting}
\end{Shaded}

\begin{verbatim}
## Setting levels: control = Failure, case = Success
\end{verbatim}

\begin{verbatim}
## Setting direction: controls < cases
\end{verbatim}

\begin{verbatim}
## 
## Call:
## roc.formula(formula = churn_train$Failure.binary ~ m1_prob, plot = TRUE,     legacy.axes = FALSE, percent = TRUE, col = "black", lwd = 2,     print.auc = TRUE)
## 
## Data: m1_prob in 104 controls (churn_train$Failure.binary Failure) < 53 cases (churn_train$Failure.binary Success).
## Area under the curve: 58.33%
\end{verbatim}

\begin{Shaded}
\begin{Highlighting}[]
\FunctionTok{plot.roc}\NormalTok{(churn\_train}\SpecialCharTok{$}\NormalTok{Failure.binary }\SpecialCharTok{\textasciitilde{}}\NormalTok{ m2\_prob,  }\AttributeTok{percent=}\ConstantTok{TRUE}\NormalTok{, }\AttributeTok{col=}\StringTok{"red"}\NormalTok{, }
         \AttributeTok{lwd=}\DecValTok{2}\NormalTok{, }\AttributeTok{print.auc=}\ConstantTok{TRUE}\NormalTok{, }\AttributeTok{add=}\ConstantTok{TRUE}\NormalTok{, }\AttributeTok{print.auc.y=}\DecValTok{40}\NormalTok{)}
\end{Highlighting}
\end{Shaded}

\begin{verbatim}
## Setting levels: control = Failure, case = Success
## Setting direction: controls < cases
\end{verbatim}

\begin{Shaded}
\begin{Highlighting}[]
\FunctionTok{plot.roc}\NormalTok{(churn\_train}\SpecialCharTok{$}\NormalTok{Failure.binary }\SpecialCharTok{\textasciitilde{}}\NormalTok{ m3\_prob,  }\AttributeTok{percent=}\ConstantTok{TRUE}\NormalTok{, }\AttributeTok{col=}\StringTok{"blue"}\NormalTok{, }
         \AttributeTok{lwd=}\DecValTok{2}\NormalTok{, }\AttributeTok{print.auc=}\ConstantTok{TRUE}\NormalTok{, }\AttributeTok{add=}\ConstantTok{TRUE}\NormalTok{, }\AttributeTok{print.auc.y=}\DecValTok{30}\NormalTok{)}
\end{Highlighting}
\end{Shaded}

\begin{verbatim}
## Setting levels: control = Failure, case = Success
## Setting direction: controls < cases
\end{verbatim}

\begin{Shaded}
\begin{Highlighting}[]
\FunctionTok{title}\NormalTok{(}\AttributeTok{main =} \StringTok{"Model Performance during Training"}\NormalTok{, }\AttributeTok{line =} \FloatTok{2.5}\NormalTok{)}
\end{Highlighting}
\end{Shaded}

\includegraphics{INFS-692---Final-Project-Model-1_files/figure-latex/unnamed-chunk-12-2.pdf}

\begin{Shaded}
\begin{Highlighting}[]
\CommentTok{\#Feature Interpretation}
\FunctionTok{vip}\NormalTok{(cv\_model3, }\AttributeTok{num\_features =} \DecValTok{20}\NormalTok{)}
\end{Highlighting}
\end{Shaded}

\includegraphics{INFS-692---Final-Project-Model-1_files/figure-latex/unnamed-chunk-12-3.pdf}

\begin{Shaded}
\begin{Highlighting}[]
\CommentTok{\# Compute predicted probabilities on test data}
\NormalTok{m1\_prob }\OtherTok{\textless{}{-}} \FunctionTok{predict}\NormalTok{(cv\_model1, churn\_test, }\AttributeTok{type =} \StringTok{"prob"}\NormalTok{)}\SpecialCharTok{$}\NormalTok{Success}
\NormalTok{m2\_prob }\OtherTok{\textless{}{-}} \FunctionTok{predict}\NormalTok{(cv\_model2, churn\_test, }\AttributeTok{type =} \StringTok{"prob"}\NormalTok{)}\SpecialCharTok{$}\NormalTok{Success}
\NormalTok{m3\_prob }\OtherTok{\textless{}{-}} \FunctionTok{predict}\NormalTok{(cv\_model3, churn\_test, }\AttributeTok{type =} \StringTok{"prob"}\NormalTok{)}\SpecialCharTok{$}\NormalTok{Success}
\end{Highlighting}
\end{Shaded}

\begin{verbatim}
## Warning in predict.lm(object, newdata, se.fit, scale = 1, type = if (type == :
## prediction from a rank-deficient fit may be misleading
\end{verbatim}

\begin{Shaded}
\begin{Highlighting}[]
\CommentTok{\# Compute AUC metrics for cv\_model1,2 and 3 }
\NormalTok{perf1 }\OtherTok{\textless{}{-}} \FunctionTok{prediction}\NormalTok{(m1\_prob, churn\_test}\SpecialCharTok{$}\NormalTok{Failure.binary) }\SpecialCharTok{\%\textgreater{}\%}
  \FunctionTok{performance}\NormalTok{(}\AttributeTok{measure =} \StringTok{"tpr"}\NormalTok{, }\AttributeTok{x.measure =} \StringTok{"fpr"}\NormalTok{)}
\NormalTok{perf2 }\OtherTok{\textless{}{-}} \FunctionTok{prediction}\NormalTok{(m2\_prob, churn\_test}\SpecialCharTok{$}\NormalTok{Failure.binary) }\SpecialCharTok{\%\textgreater{}\%}
  \FunctionTok{performance}\NormalTok{(}\AttributeTok{measure =} \StringTok{"tpr"}\NormalTok{, }\AttributeTok{x.measure =} \StringTok{"fpr"}\NormalTok{)}
\NormalTok{perf3 }\OtherTok{\textless{}{-}} \FunctionTok{prediction}\NormalTok{(m3\_prob, churn\_test}\SpecialCharTok{$}\NormalTok{Failure.binary) }\SpecialCharTok{\%\textgreater{}\%}
  \FunctionTok{performance}\NormalTok{(}\AttributeTok{measure =} \StringTok{"tpr"}\NormalTok{, }\AttributeTok{x.measure =} \StringTok{"fpr"}\NormalTok{)}

\CommentTok{\# Plot ROC curves for cv\_model1,2 and 3 }
\FunctionTok{plot}\NormalTok{(perf1, }\AttributeTok{col =} \StringTok{"black"}\NormalTok{, }\AttributeTok{print.auc=}\ConstantTok{TRUE}\NormalTok{, }\AttributeTok{lty =} \DecValTok{2}\NormalTok{)}
\FunctionTok{plot}\NormalTok{(perf2,  }\AttributeTok{add =} \ConstantTok{TRUE}\NormalTok{, }\AttributeTok{col =} \StringTok{"red"}\NormalTok{,  }\AttributeTok{print.auc=}\ConstantTok{TRUE}\NormalTok{, }\AttributeTok{lty =} \DecValTok{2}\NormalTok{)}
\FunctionTok{plot}\NormalTok{(perf3, }\AttributeTok{add =} \ConstantTok{TRUE}\NormalTok{, }\AttributeTok{col =} \StringTok{"blue"}\NormalTok{, }\AttributeTok{print.auc=}\ConstantTok{TRUE}\NormalTok{)}
\FunctionTok{legend}\NormalTok{(}\FloatTok{0.7}\NormalTok{, }\FloatTok{0.3}\NormalTok{, }\AttributeTok{legend =} \FunctionTok{c}\NormalTok{(}\StringTok{"cv\_model1"}\NormalTok{, }\StringTok{"cv\_model2"}\NormalTok{, }\StringTok{"cv\_model3"}\NormalTok{),}
       \AttributeTok{col =} \FunctionTok{c}\NormalTok{(}\StringTok{"black"}\NormalTok{,}\StringTok{"red"}\NormalTok{, }\StringTok{"blue"}\NormalTok{), }\AttributeTok{lty =} \DecValTok{3}\SpecialCharTok{:}\DecValTok{1}\NormalTok{, }\AttributeTok{cex =} \FloatTok{0.6}\NormalTok{)}
\end{Highlighting}
\end{Shaded}

\includegraphics{INFS-692---Final-Project-Model-1_files/figure-latex/unnamed-chunk-12-4.pdf}

\begin{Shaded}
\begin{Highlighting}[]
\CommentTok{\# ROC plot for testing data}
\FunctionTok{roc}\NormalTok{(churn\_test}\SpecialCharTok{$}\NormalTok{Failure.binary }\SpecialCharTok{\textasciitilde{}}\NormalTok{ m1\_prob, }\AttributeTok{plot=}\ConstantTok{TRUE}\NormalTok{, }\AttributeTok{legacy.axes=}\ConstantTok{FALSE}\NormalTok{, }
    \AttributeTok{percent=}\ConstantTok{TRUE}\NormalTok{, }\AttributeTok{col=}\StringTok{"black"}\NormalTok{, }\AttributeTok{lwd=}\DecValTok{2}\NormalTok{, }\AttributeTok{print.auc=}\ConstantTok{TRUE}\NormalTok{)}
\end{Highlighting}
\end{Shaded}

\begin{verbatim}
## Setting levels: control = Failure, case = Success
## Setting direction: controls < cases
\end{verbatim}

\begin{verbatim}
## 
## Call:
## roc.formula(formula = churn_test$Failure.binary ~ m1_prob, plot = TRUE,     legacy.axes = FALSE, percent = TRUE, col = "black", lwd = 2,     print.auc = TRUE)
## 
## Data: m1_prob in 26 controls (churn_test$Failure.binary Failure) < 14 cases (churn_test$Failure.binary Success).
## Area under the curve: 48.35%
\end{verbatim}

\begin{Shaded}
\begin{Highlighting}[]
\FunctionTok{plot.roc}\NormalTok{(churn\_test}\SpecialCharTok{$}\NormalTok{Failure.binary }\SpecialCharTok{\textasciitilde{}}\NormalTok{ m2\_prob,  }\AttributeTok{percent=}\ConstantTok{TRUE}\NormalTok{, }\AttributeTok{col=}\StringTok{"red"}\NormalTok{, }
         \AttributeTok{lwd=}\DecValTok{2}\NormalTok{, }\AttributeTok{print.auc=}\ConstantTok{TRUE}\NormalTok{, }\AttributeTok{add=}\ConstantTok{TRUE}\NormalTok{, }\AttributeTok{print.auc.y=}\DecValTok{40}\NormalTok{)}
\end{Highlighting}
\end{Shaded}

\begin{verbatim}
## Setting levels: control = Failure, case = Success
## Setting direction: controls < cases
\end{verbatim}

\begin{Shaded}
\begin{Highlighting}[]
\FunctionTok{plot.roc}\NormalTok{(churn\_test}\SpecialCharTok{$}\NormalTok{Failure.binary }\SpecialCharTok{\textasciitilde{}}\NormalTok{ m3\_prob,  }\AttributeTok{percent=}\ConstantTok{TRUE}\NormalTok{, }\AttributeTok{col=}\StringTok{"blue"}\NormalTok{, }
         \AttributeTok{lwd=}\DecValTok{2}\NormalTok{, }\AttributeTok{print.auc=}\ConstantTok{TRUE}\NormalTok{, }\AttributeTok{add=}\ConstantTok{TRUE}\NormalTok{, }\AttributeTok{print.auc.y=}\DecValTok{30}\NormalTok{)}
\end{Highlighting}
\end{Shaded}

\begin{verbatim}
## Setting levels: control = Failure, case = Success
## Setting direction: controls < cases
\end{verbatim}

\begin{Shaded}
\begin{Highlighting}[]
\FunctionTok{title}\NormalTok{(}\AttributeTok{main =} \StringTok{"Model Performance during Testing"}\NormalTok{, }\AttributeTok{line =} \FloatTok{2.5}\NormalTok{)}


\FunctionTok{legend}\NormalTok{(}\StringTok{"topright"}\NormalTok{, }\FunctionTok{c}\NormalTok{(}\StringTok{"Model 1"}\NormalTok{, }\StringTok{"Model 2"}\NormalTok{, }\StringTok{"Model 3"}\NormalTok{), }\AttributeTok{lty=}\DecValTok{1}\NormalTok{, }
    \AttributeTok{col =} \FunctionTok{c}\NormalTok{(}\StringTok{"black"}\NormalTok{, }\StringTok{"red"}\NormalTok{, }\StringTok{"blue"}\NormalTok{), }\AttributeTok{bty=}\StringTok{"n"}\NormalTok{)}
\end{Highlighting}
\end{Shaded}

\includegraphics{INFS-692---Final-Project-Model-1_files/figure-latex/unnamed-chunk-12-5.pdf}

As seen in the ROC graph, Model 2 performs in an outstancing manner of
distinguishing failures and successes, whereas model 3, in which
Failure.binary as a whole performs on the entire dataset, still performs
at an excellent level. Model 1, technically, shouldn't really be
considered.

\hypertarget{model-2-knn}{%
\subsubsection{Model 2: KNN}\label{model-2-knn}}

Here we are creating a second model using KNN. The grid search on my
computer takes about 5-7 minutes with a sample size of 50 variables.
Again, reference points for prediction is ``Success''.

\begin{Shaded}
\begin{Highlighting}[]
\CommentTok{\#{-}{-}{-}{-}{-}{-}Blueprint{-}{-}{-}{-}{-}{-}{-}{-}{-}{-}{-}{-}{-}{-}{-}{-}{-}{-}{-}{-}{-}{-}{-}{-}{-}{-}{-}{-}{-}{-}{-}{-}{-}{-}{-}{-}{-}{-}{-}{-}{-}{-}{-}\#}

\NormalTok{blueprint\_attr }\OtherTok{\textless{}{-}} \FunctionTok{recipe}\NormalTok{(Failure.binary }\SpecialCharTok{\textasciitilde{}}\NormalTok{ ., }\AttributeTok{data =}\NormalTok{ churn\_train) }\SpecialCharTok{\%\textgreater{}\%}
  \FunctionTok{step\_nzv}\NormalTok{(}\FunctionTok{all\_nominal}\NormalTok{()) }\SpecialCharTok{\%\textgreater{}\%}
  \FunctionTok{step\_integer}\NormalTok{(}\FunctionTok{contains}\NormalTok{(}\StringTok{"Entropy"}\NormalTok{)) }\SpecialCharTok{\%\textgreater{}\%}
  \FunctionTok{step\_integer}\NormalTok{(}\FunctionTok{contains}\NormalTok{(}\StringTok{"GLNU"}\NormalTok{)) }\SpecialCharTok{\%\textgreater{}\%}
  \FunctionTok{step\_dummy}\NormalTok{(}\FunctionTok{all\_nominal}\NormalTok{(), }\SpecialCharTok{{-}}\FunctionTok{all\_outcomes}\NormalTok{(), }\AttributeTok{one\_hot =} \ConstantTok{TRUE}\NormalTok{) }\SpecialCharTok{\%\textgreater{}\%}
  \FunctionTok{step\_center}\NormalTok{(}\FunctionTok{all\_numeric}\NormalTok{(), }\SpecialCharTok{{-}}\FunctionTok{all\_outcomes}\NormalTok{()) }\SpecialCharTok{\%\textgreater{}\%}
  \FunctionTok{step\_scale}\NormalTok{(}\FunctionTok{all\_numeric}\NormalTok{(), }\SpecialCharTok{{-}}\FunctionTok{all\_outcomes}\NormalTok{())}

\CommentTok{\#{-}{-}{-}{-}{-}{-}{-}{-}{-}{-}{-}Resampling Method{-}{-}{-}{-}{-}{-}{-}{-}{-}{-}{-}{-}{-}{-}{-}{-}{-}{-}{-}{-}{-}{-}{-}{-}{-}{-}{-}{-}\#}

\NormalTok{cv }\OtherTok{\textless{}{-}} \FunctionTok{trainControl}\NormalTok{(}
  \AttributeTok{method =} \StringTok{"repeatedcv"}\NormalTok{, }
  \AttributeTok{number =} \DecValTok{10}\NormalTok{, }
  \AttributeTok{repeats =} \DecValTok{5}\NormalTok{,}
  \AttributeTok{classProbs =} \ConstantTok{TRUE}\NormalTok{,                 }
  \AttributeTok{summaryFunction =}\NormalTok{ twoClassSummary)}
\CommentTok{\#{-}{-}{-}{-}{-}{-}{-}Hyperparameters and Gridsearch{-}{-}{-}{-}{-}{-}{-}{-}{-}{-}{-}{-}{-}{-}{-}{-}{-}{-}{-}{-}{-}{-}{-}{-}{-}{-}{-}\#}

\NormalTok{hyper\_grid }\OtherTok{\textless{}{-}} \FunctionTok{expand.grid}\NormalTok{(}
  \AttributeTok{k =} \FunctionTok{floor}\NormalTok{(}\FunctionTok{seq}\NormalTok{(}\DecValTok{1}\NormalTok{, }\FunctionTok{nrow}\NormalTok{(churn\_train)}\SpecialCharTok{/}\DecValTok{3}\NormalTok{, }\AttributeTok{length.out =} \DecValTok{20}\NormalTok{))}
\NormalTok{)}

\CommentTok{\# Fit knn model and perform grid search}
\NormalTok{knn\_grid }\OtherTok{\textless{}{-}} \FunctionTok{train}\NormalTok{(}
\NormalTok{  blueprint\_attr, }
  \AttributeTok{data =}\NormalTok{ churn\_train, }
  \AttributeTok{method =} \StringTok{"knn"}\NormalTok{, }
  \AttributeTok{trControl =}\NormalTok{ cv, }
  \AttributeTok{tuneGrid =}\NormalTok{ hyper\_grid,}
  \AttributeTok{metric =} \StringTok{"ROC"}
\NormalTok{)}

\FunctionTok{ggplot}\NormalTok{(knn\_grid)}
\end{Highlighting}
\end{Shaded}

\includegraphics{INFS-692---Final-Project-Model-1_files/figure-latex/unnamed-chunk-13-1.pdf}

\begin{Shaded}
\begin{Highlighting}[]
\CommentTok{\#{-}{-}{-}{-}{-}{-}{-}{-}{-}{-}Variable Importance{-}{-}{-}{-}{-}{-}{-}{-}{-}{-}{-}{-}{-}{-}{-}{-}\#}

\NormalTok{varimpo }\OtherTok{\textless{}{-}} \FunctionTok{varImp}\NormalTok{(knn\_grid)}
\NormalTok{varimpo}
\end{Highlighting}
\end{Shaded}

\begin{verbatim}
## ROC curve variable importance
## 
##   only 20 most important variables shown (out of 49)
## 
##                             Importance
## Entropy_cooc.W.ADC              100.00
## GLNU_align.H.PET                 54.91
## DVAR_cooc.L.PET                  43.14
## Contrast_cooc.L.PET              32.67
## Compactness_v2.PET               29.76
## DAVE_cooc.L.PET                  28.52
## Dissimilarity_cooc.L.PET         28.52
## DENT_cooc.L.PET                  27.69
## Variance_cooc.L.PET              26.43
## Sphericity.PET                   26.35
## Entropy_hist.PET                 22.03
## Compactness_v1.PET               19.20
## H_suv.PET                        19.02
## SVAR_cooc.L.PET                  18.18
## tumor_length.PET                 18.01
## SAVE_cooc.L.PET                  17.93
## Average_cooc.L.PET               17.84
## Center_of_mass.PET               14.99
## Spherical_disproportion.PET      12.98
## ratio_3ds_vol_norm.PET           12.98
\end{verbatim}

\begin{Shaded}
\begin{Highlighting}[]
\NormalTok{pred\_knngrid }\OtherTok{\textless{}{-}} \FunctionTok{predict}\NormalTok{(knn\_grid, churn\_train)}

\FunctionTok{confusionMatrix}\NormalTok{(}
  \AttributeTok{data =} \FunctionTok{relevel}\NormalTok{(pred\_knngrid, }\AttributeTok{ref =} \StringTok{"Success"}\NormalTok{), }
  \AttributeTok{reference =} \FunctionTok{relevel}\NormalTok{(churn\_train}\SpecialCharTok{$}\NormalTok{Failure.binary, }\AttributeTok{ref =} \StringTok{"Success"}\NormalTok{)}
\NormalTok{)}
\end{Highlighting}
\end{Shaded}

\begin{verbatim}
## Confusion Matrix and Statistics
## 
##           Reference
## Prediction Success Failure
##    Success      18       4
##    Failure      35     100
##                                          
##                Accuracy : 0.7516         
##                  95% CI : (0.6764, 0.817)
##     No Information Rate : 0.6624         
##     P-Value [Acc > NIR] : 0.01003        
##                                          
##                   Kappa : 0.3516         
##                                          
##  Mcnemar's Test P-Value : 1.556e-06      
##                                          
##             Sensitivity : 0.3396         
##             Specificity : 0.9615         
##          Pos Pred Value : 0.8182         
##          Neg Pred Value : 0.7407         
##              Prevalence : 0.3376         
##          Detection Rate : 0.1146         
##    Detection Prevalence : 0.1401         
##       Balanced Accuracy : 0.6506         
##                                          
##        'Positive' Class : Success        
## 
\end{verbatim}

\begin{Shaded}
\begin{Highlighting}[]
\FunctionTok{ggplot}\NormalTok{(varimpo)}
\end{Highlighting}
\end{Shaded}

\includegraphics{INFS-692---Final-Project-Model-1_files/figure-latex/unnamed-chunk-13-2.pdf}

\begin{Shaded}
\begin{Highlighting}[]
\FunctionTok{par}\NormalTok{(}\AttributeTok{mfrow =} \FunctionTok{c}\NormalTok{(}\DecValTok{1}\NormalTok{,}\DecValTok{2}\NormalTok{))}



\CommentTok{\# § Plot the training data performance while print the AUC values.}


\NormalTok{knngrid\_prob }\OtherTok{\textless{}{-}} \FunctionTok{predict}\NormalTok{(knn\_grid, churn\_train, }\AttributeTok{type =} \StringTok{"prob"}\NormalTok{)}\SpecialCharTok{$}\NormalTok{Success}
\FunctionTok{roc}\NormalTok{(churn\_train}\SpecialCharTok{$}\NormalTok{Failure.binary }\SpecialCharTok{\textasciitilde{}}\NormalTok{ knngrid\_prob, }\AttributeTok{plot=}\ConstantTok{TRUE}\NormalTok{, }\AttributeTok{legacy.axes=}\ConstantTok{FALSE}\NormalTok{, }
    \AttributeTok{percent=}\ConstantTok{TRUE}\NormalTok{, }\AttributeTok{col=}\StringTok{"black"}\NormalTok{, }\AttributeTok{lwd=}\DecValTok{2}\NormalTok{, }\AttributeTok{print.auc=}\ConstantTok{TRUE}\NormalTok{)}
\end{Highlighting}
\end{Shaded}

\begin{verbatim}
## Setting levels: control = Failure, case = Success
\end{verbatim}

\begin{verbatim}
## Setting direction: controls < cases
\end{verbatim}

\begin{verbatim}
## 
## Call:
## roc.formula(formula = churn_train$Failure.binary ~ knngrid_prob,     plot = TRUE, legacy.axes = FALSE, percent = TRUE, col = "black",     lwd = 2, print.auc = TRUE)
## 
## Data: knngrid_prob in 104 controls (churn_train$Failure.binary Failure) < 53 cases (churn_train$Failure.binary Success).
## Area under the curve: 79.23%
\end{verbatim}

\begin{Shaded}
\begin{Highlighting}[]
\FunctionTok{title}\NormalTok{(}\AttributeTok{main =} \StringTok{"Model Performance during Training"}\NormalTok{, }\AttributeTok{line =} \FloatTok{2.5}\NormalTok{)}


\CommentTok{\# § Use the PREDICT function to predict using the testing data.}

\NormalTok{knntest }\OtherTok{\textless{}{-}} \FunctionTok{predict}\NormalTok{(knn\_grid, churn\_test)}
\FunctionTok{confusionMatrix}\NormalTok{(}
  \AttributeTok{data =} \FunctionTok{relevel}\NormalTok{(knntest, }\AttributeTok{ref =} \StringTok{"Success"}\NormalTok{), }
  \AttributeTok{reference =} \FunctionTok{relevel}\NormalTok{(churn\_test}\SpecialCharTok{$}\NormalTok{Failure.binary, }\AttributeTok{ref =} \StringTok{"Success"}\NormalTok{)}
\NormalTok{)}
\end{Highlighting}
\end{Shaded}

\begin{verbatim}
## Confusion Matrix and Statistics
## 
##           Reference
## Prediction Success Failure
##    Success       0       0
##    Failure      14      26
##                                           
##                Accuracy : 0.65            
##                  95% CI : (0.4832, 0.7937)
##     No Information Rate : 0.65            
##     P-Value [Acc > NIR] : 0.572082        
##                                           
##                   Kappa : 0               
##                                           
##  Mcnemar's Test P-Value : 0.000512        
##                                           
##             Sensitivity : 0.00            
##             Specificity : 1.00            
##          Pos Pred Value :  NaN            
##          Neg Pred Value : 0.65            
##              Prevalence : 0.35            
##          Detection Rate : 0.00            
##    Detection Prevalence : 0.00            
##       Balanced Accuracy : 0.50            
##                                           
##        'Positive' Class : Success         
## 
\end{verbatim}

\begin{Shaded}
\begin{Highlighting}[]
\CommentTok{\# § Plot the testing data performance while print the AUC values.}

\NormalTok{knngrid\_probtest }\OtherTok{\textless{}{-}} \FunctionTok{predict}\NormalTok{(knn\_grid, churn\_test, }\AttributeTok{type =} \StringTok{"prob"}\NormalTok{)}\SpecialCharTok{$}\NormalTok{Success}
\FunctionTok{roc}\NormalTok{(churn\_test}\SpecialCharTok{$}\NormalTok{Failure.binary }\SpecialCharTok{\textasciitilde{}}\NormalTok{ knngrid\_probtest, }\AttributeTok{plot=}\ConstantTok{TRUE}\NormalTok{, }\AttributeTok{legacy.axes=}\ConstantTok{FALSE}\NormalTok{, }
    \AttributeTok{percent=}\ConstantTok{TRUE}\NormalTok{, }\AttributeTok{col=}\StringTok{"black"}\NormalTok{, }\AttributeTok{lwd=}\DecValTok{2}\NormalTok{, }\AttributeTok{print.auc=}\ConstantTok{TRUE}\NormalTok{)}
\end{Highlighting}
\end{Shaded}

\begin{verbatim}
## Setting levels: control = Failure, case = Success
## Setting direction: controls < cases
\end{verbatim}

\begin{verbatim}
## 
## Call:
## roc.formula(formula = churn_test$Failure.binary ~ knngrid_probtest,     plot = TRUE, legacy.axes = FALSE, percent = TRUE, col = "black",     lwd = 2, print.auc = TRUE)
## 
## Data: knngrid_probtest in 26 controls (churn_test$Failure.binary Failure) < 14 cases (churn_test$Failure.binary Success).
## Area under the curve: 71.02%
\end{verbatim}

\begin{Shaded}
\begin{Highlighting}[]
\FunctionTok{title}\NormalTok{(}\AttributeTok{main =} \StringTok{"Model Performance during Testing"}\NormalTok{, }\AttributeTok{line =} \FloatTok{2.5}\NormalTok{)}
\end{Highlighting}
\end{Shaded}

\includegraphics{INFS-692---Final-Project-Model-1_files/figure-latex/unnamed-chunk-13-3.pdf}
The training data performance is just a little better than testing, but
the AUC values for both aren't as good as the ones from LR.

\hypertarget{model-3-decision-tree}{%
\subsubsection{Model 3: Decision Tree}\label{model-3-decision-tree}}

\begin{Shaded}
\begin{Highlighting}[]
\DocumentationTok{\#\#modeling}
\NormalTok{fit }\OtherTok{\textless{}{-}} \FunctionTok{rpart}\NormalTok{(Failure.binary}\SpecialCharTok{\textasciitilde{}}\NormalTok{., }\AttributeTok{data =}\NormalTok{ churn\_train, }\AttributeTok{method =} \StringTok{\textquotesingle{}class\textquotesingle{}}\NormalTok{)}

\CommentTok{\#plotting}
\FunctionTok{rpart.plot}\NormalTok{(fit, }\AttributeTok{extra =} \DecValTok{100}\NormalTok{)}
\end{Highlighting}
\end{Shaded}

\includegraphics{INFS-692---Final-Project-Model-1_files/figure-latex/unnamed-chunk-14-1.pdf}

\begin{Shaded}
\begin{Highlighting}[]
\CommentTok{\#plotting}
\FunctionTok{plotcp}\NormalTok{(fit)}
\end{Highlighting}
\end{Shaded}

\includegraphics{INFS-692---Final-Project-Model-1_files/figure-latex/unnamed-chunk-14-2.pdf}

\begin{Shaded}
\begin{Highlighting}[]
\CommentTok{\#feature importance}
\FunctionTok{vip}\NormalTok{(fit, }\AttributeTok{num\_features =} \DecValTok{20}\NormalTok{, }\AttributeTok{bar =} \ConstantTok{FALSE}\NormalTok{)}
\end{Highlighting}
\end{Shaded}

\includegraphics{INFS-692---Final-Project-Model-1_files/figure-latex/unnamed-chunk-14-3.pdf}

\begin{Shaded}
\begin{Highlighting}[]
\CommentTok{\# Compute predicted probabilities on training data}
\NormalTok{dt1\_prob }\OtherTok{\textless{}{-}} \FunctionTok{predict}\NormalTok{(fit, churn\_train, }\AttributeTok{type =} \StringTok{"prob"}\NormalTok{)}

\CommentTok{\# ROC plot for training data}
\FunctionTok{roc}\NormalTok{(churn\_train}\SpecialCharTok{$}\NormalTok{Failure.binary }\SpecialCharTok{\textasciitilde{}}\NormalTok{ dt1\_prob[,}\DecValTok{2}\NormalTok{], }\AttributeTok{plot=}\ConstantTok{TRUE}\NormalTok{, }\AttributeTok{legacy.axes=}\ConstantTok{FALSE}\NormalTok{, }
    \AttributeTok{percent=}\ConstantTok{TRUE}\NormalTok{, }\AttributeTok{col=}\StringTok{"black"}\NormalTok{, }\AttributeTok{lwd=}\DecValTok{2}\NormalTok{, }\AttributeTok{print.auc=}\ConstantTok{TRUE}\NormalTok{)}
\end{Highlighting}
\end{Shaded}

\begin{verbatim}
## Setting levels: control = Failure, case = Success
\end{verbatim}

\begin{verbatim}
## Setting direction: controls < cases
\end{verbatim}

\includegraphics{INFS-692---Final-Project-Model-1_files/figure-latex/unnamed-chunk-14-4.pdf}

\begin{verbatim}
## 
## Call:
## roc.formula(formula = churn_train$Failure.binary ~ dt1_prob[,     2], plot = TRUE, legacy.axes = FALSE, percent = TRUE, col = "black",     lwd = 2, print.auc = TRUE)
## 
## Data: dt1_prob[, 2] in 104 controls (churn_train$Failure.binary Failure) < 53 cases (churn_train$Failure.binary Success).
## Area under the curve: 90.97%
\end{verbatim}

\begin{Shaded}
\begin{Highlighting}[]
\NormalTok{test\_fit }\OtherTok{\textless{}{-}} \FunctionTok{rpart}\NormalTok{(Failure.binary}\SpecialCharTok{\textasciitilde{}}\NormalTok{., }\AttributeTok{data =}\NormalTok{ churn\_test, }\AttributeTok{method =} \StringTok{\textquotesingle{}class\textquotesingle{}}\NormalTok{)}
\NormalTok{failure\_predict }\OtherTok{\textless{}{-}} \FunctionTok{predict}\NormalTok{(test\_fit,churn\_test)}


\CommentTok{\#    Use the RPART.PLOT and PLOTCP function to identify the trees}
\FunctionTok{rpart.plot}\NormalTok{(test\_fit, }\AttributeTok{extra =}  \DecValTok{100}\NormalTok{)}
\end{Highlighting}
\end{Shaded}

\includegraphics{INFS-692---Final-Project-Model-1_files/figure-latex/unnamed-chunk-14-5.pdf}

\begin{Shaded}
\begin{Highlighting}[]
\FunctionTok{plotcp}\NormalTok{(test\_fit)}
\end{Highlighting}
\end{Shaded}

\includegraphics{INFS-692---Final-Project-Model-1_files/figure-latex/unnamed-chunk-14-6.pdf}

\begin{Shaded}
\begin{Highlighting}[]
\CommentTok{\#    Plot the testing data performance while print the AUC values}


\NormalTok{dt2\_prob }\OtherTok{\textless{}{-}} \FunctionTok{predict}\NormalTok{(test\_fit, churn\_test, }\AttributeTok{type =} \StringTok{"prob"}\NormalTok{)}

\NormalTok{perf1 }\OtherTok{\textless{}{-}} \FunctionTok{prediction}\NormalTok{(dt2\_prob[,}\DecValTok{2}\NormalTok{], churn\_test}\SpecialCharTok{$}\NormalTok{Failure.binary) }\SpecialCharTok{\%\textgreater{}\%}
  \FunctionTok{performance}\NormalTok{(}\AttributeTok{measure =} \StringTok{"tpr"}\NormalTok{, }\AttributeTok{x.measure =} \StringTok{"fpr"}\NormalTok{)}
\FunctionTok{plot}\NormalTok{(perf1, }\AttributeTok{col =} \StringTok{"black"}\NormalTok{, }\AttributeTok{lty =} \DecValTok{2}\NormalTok{)}
\end{Highlighting}
\end{Shaded}

\includegraphics{INFS-692---Final-Project-Model-1_files/figure-latex/unnamed-chunk-14-7.pdf}

\begin{Shaded}
\begin{Highlighting}[]
\FunctionTok{roc}\NormalTok{(churn\_test}\SpecialCharTok{$}\NormalTok{Failure.binary }\SpecialCharTok{\textasciitilde{}}\NormalTok{ dt2\_prob[,}\DecValTok{2}\NormalTok{], }\AttributeTok{plot=}\ConstantTok{TRUE}\NormalTok{, }\AttributeTok{legacy.axes=}\ConstantTok{FALSE}\NormalTok{, }
    \AttributeTok{percent=}\ConstantTok{TRUE}\NormalTok{, }\AttributeTok{col=}\StringTok{"black"}\NormalTok{, }\AttributeTok{lwd=}\DecValTok{2}\NormalTok{, }\AttributeTok{print.auc=}\ConstantTok{TRUE}\NormalTok{)}
\end{Highlighting}
\end{Shaded}

\begin{verbatim}
## Setting levels: control = Failure, case = Success
## Setting direction: controls < cases
\end{verbatim}

\includegraphics{INFS-692---Final-Project-Model-1_files/figure-latex/unnamed-chunk-14-8.pdf}

\begin{verbatim}
## 
## Call:
## roc.formula(formula = churn_test$Failure.binary ~ dt2_prob[,     2], plot = TRUE, legacy.axes = FALSE, percent = TRUE, col = "black",     lwd = 2, print.auc = TRUE)
## 
## Data: dt2_prob[, 2] in 26 controls (churn_test$Failure.binary Failure) < 14 cases (churn_test$Failure.binary Success).
## Area under the curve: 90.93%
\end{verbatim}

The Decision Tree ROC curves and AUC values are excellent considering
both training and testing are over 90\%.

\hypertarget{conclusion}{%
\subsubsection{Conclusion}\label{conclusion}}

Based on the above AUC results, KNN has performed the least well
compared to both Linear Regression and Decision Tree. If we consider the
ROC models that had Failure.binary be predicted against the full
(sampled) dataset, Decision Tree had the highest accuracy.

\end{document}
